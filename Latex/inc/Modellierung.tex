\chapter{Modellbildung und Regelung}

\section{Model}
F�r die Modellbidung wird der dreidimensionale Ballbot in drei unabh�ngige planare Modelle aufgeteilt. In jeder Ebene wird das System vereinfacht als Zusammensetzung von drei starren K�rpern bestehend aus einer Kugel, ein virtuelles Rad und einen K�rper betrachtet und besitzt somit zwei Freiheitsgrade, die sich in eine Translation bzw. Rotation des Balles und eine Rotation des K�rpers aufteilen lassen\cite{Manual}.\newline
Um ein m�glichst vereinfachte Modelle der drei Ebenen zu erhalten, werden weitere Annahmen getroffen, die im Folgenden erl�utert werden: 
\begin{itemize}
	\item Kein Schlupf: Das System besitzt zwei Kontaktpunkte, in denen ein Schlupf auftreten kann. Hierzu z�hlt zum einen der Kontaktpunkt von Ball und Boden und zum zweiten der Kontaktpunkt zwischen den R�dern und dem Ball. Damit dies gew�hrleistet wird, m�ssen die angelegten Drehmomente begrenzt werden.
	\item Keine Reibung: Der einzige Vorgang im System, bei dem die Reibung nicht vernachl�ssigt wird, ist bei der Rotation des Balles um die $z$-Achse. Bei den anderen Vorg�ngen, bei der in der Realit�t auch Reibung auftritt, wird im Modell vernachl�ssigt.
	\item Keine Deformation: Bei dem eingesetzten Ball handelt es sich nicht um eine hohle Stahlkugel, sondern um ein elastischen, mit Luft bef�llbaren Ball. Deshalb wird die Deformation des Balles in der Modellbeschreibung nicht mit einbezogen, um die Komplexit�t gering zu halten. 
	\item Schnelle Motorendynamik: F�r die Gleichgewichtsstabilisierung des Roboters ist es wichtig, dass die Motoren eine schneller Dynamik als das System aufweisen. 
	\item Horizontale Bewegung: Das System wird f�r die horizontale Bewegung auf einer flachen Oberfl�che ohne starken Neigungen ausgelegt. Somit wird die vertikale Bewegung vernachl�ssigt. 
\end{itemize}
Mit den getroffenen Annahmen ist es m�glich, das Modell des Ballbots aufzustellen. 

\section{Energien}
F�r das Aufstellen der Bewegungsgleichungen des Systems wird der Lagrange Ansatz angewendet. Dazu m�ssen im Voraus die potentiellen und kinetischen Energien der einzelnen K�rper aufgestellt werden. Dabei ist darauf zu achten,dass die Formeln der potentiellen und kinetischen Energien f�r die $yz$- und der $xz$-Ebene identisch sind und dagegen sich sie die Energie $xy$-Ebene unterscheiden. Deshalb werden im Folgenden die Formeln der $yz$ und $xy$-Ebenen angegeben.\newline
Zun�chst werden die kinetischen und potentiellen Energien des Balles in den entsprechenden Ebenen betrachten. 
F�r die kinetische Energie werden folgenden Formeln der jeweiligen Ebenen angegeben.
\begin{equation}
	T_{S,yz} = \frac{1}{2} \cdot m_{S} \cdot (r_{S} \cdot \dot \varphi_{x})^2 + \frac{1}{2} \cdot I_{S} \cdot \dot\varphi_{x}^2
\end{equation}
\begin{equation}
	T_{S,xy} =\frac{1}{2} \cdot I_{S} \cdot \dot\varphi_{z}^2
\end{equation}
Da der Ursprung des Weltkoordinatensystems in den Mittelpunkt des Balles gelegt wird, besitzt das System sowohl in der $yz$ als auch in der $yx$ Ebene keine potentielle Energie. 
\begin{equation}
	V_{S,yz}=0
\end{equation}

Die kinetischen Energien des virtuellen Rades der jeweiligen Ebene werden mit den folgenden Formeln angegeben.
\begin{equation}
	T_{W,yz} = \frac{1}{2} \cdot m_{W} \cdot ((r_{S} \cdot \dot \varphi_{x})^2 + 2 \cdot (r_{S}+r_{W}) \cdot \cos(\theta_{x})\cdot \dot \theta_{x} \cdot (r_{S}\cdot\dot\varphi_{x})+(r_{S} + r_{W})^2\cdot\dot\theta_{x}^2) + \frac{1}{2} \cdot I_{W}\cdot(\frac{r_{S}}{r_{W}}\cdot(\dot\varphi_{x}-\dot\theta_{x})-\dot\theta_{x})^2
\end{equation}
\begin{equation}
	T_{W,xy} =\frac{1}{2} \cdot I_{W} \cdot \dot\Psi_{z}^2
\end{equation}

Die potentielle Energie des virtuellen Rades in der $yz$ Ebene kann folgenderma�en berechnet werden.
\begin{equation}
V_{W,yz} = m_{W} \cdot g \cdot (r_{S} + r_{W})\cdot \cos(\theta_{x})
\end{equation}

Die kinetischen Energien f�r die $yz$ und der $xy$ Ebene f�r den Roboterk�rper sind �hnlich der des virtuellen Rades und werden folgenderma�en berechnet. 
\begin{equation}
T_{B,yz} = \frac{1}{2} \cdot m_{A} \cdot ((r_{S} \cdot \dot \varphi_{x})^2 + 2 \cdot l \cdot \cos(\theta_{x})\cdot \dot \theta_{x} \cdot (r_{S}\cdot\dot\varphi_{x}) + l^2\cdot\dot\theta_{x}^2)+\frac{1}{2} \cdot I_{A}\cdot\dot\theta_{x}^2
\end{equation}
\begin{equation}
	T_{B,xy} =\frac{1}{2} \cdot I_{W,xy} \cdot \dot\Psi_{z}^2
\end{equation}

Auch die dazugeh�rige potentielle Energie �hnelt der des virtuellen Rades. Der Unterschied liegt zum einem im Gewicht des K�rpers $m_{B}$ und der H�he $l$ von Ballmittelpunkt zum Schwerpunkt des Roboteraufbaus.
\begin{equation}
V_{B,yz} = m_{A} \cdot g \cdot l \cdot \cos(\theta_{x})
\end{equation}


