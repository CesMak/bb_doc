\section*{Kurzfassung}
Inhalt dieser Arbeit ist der mechanische Aufbau, die Modellierung, der Reglerentwurf und die Simulation eines Ballbots. Aufgebaut wurde der Ballbot mittels Komponenten eines Turtlebot3 Roboters, der um einen selbst entwickelten Unterbau erg�nzt wurde. Anschlie�end erfolgte die mathematische Modellbildung und eine linear-quadratische Reglerauslegung(LQR) in MATLAB/Simulink. Bei der Modellbildung wurde das reale dreidimensionale System durch zwei unabh�ngige planare Ebenen ($xz$ und $yz$) approximiert. Der entworfene Regler wurde schie�lich mit dem 3D-Simulator Gazebo validiert. 

Es konnte gezeigt werden, dass sich das System mittels der entworfenen Regelung in der $xz$- und $yz$-Ebene stabilisieren l�sst. Essentiell f�r das Balancieren des Ballbots ist eine optimale Abstimmung des Roboters auf den Ball. Es sollte darauf geachtet werden dass:
\begin{itemize}
	\item Der Reibkoeffizient im Kontaktpunkt zwischen Boden und Ball, sowie zwischen Ball und omnidirekionalem Rad muss m�glichst gro� sein.
	\item Die Motoren eine hohe maximale Drehzahl($>4.5$\,rad/s) aufweisen.
	\item Die Abtastrate des Systems m�glichst hoch($>130$\,Hz) ist.
\end{itemize}

Weiterhin kann das System durch eine genauere Modellierung (3D-Modellierung) sowie der Ber�cksichtigung der Ball-Odometrie stabilisiert werden.

\textbf{Schl�sselw�rter:} Ballbot, Omnidirektionale R�der, Gazebo, Matlab, Arduino


\newpage
\selectlanguage{english}
\section*{Abstract}
Goal of this research project was the construction, modelling, 
controller implementation and the simulation of a Ballbot. This Ballbot was built using components of a Turtlebot3 Robot. Additionally a specific designed strucutre was created to attach the omnidirectional wheels to the ballbot. After that MATLAB/Simulink was used to design a linear-quadratic regulator(LQR). The modelling was done by approximating the real three dimensional System with two planes (xz and yz). At the end the Ballbot was validated with the 3D-Simulator Gazebo.

This project revealed that the system could be stabilized around the xz- and the yz-plane. Essential for the balancing behaviour of the ballbot was an optimal adaption of the Robot to the ball. Thereby it should be considered:
\begin{itemize}
	\item The friction coefficient between the ball and the surface and between the ball and the omnidirectional wheel should be as high as possible.
	\item The maximum revolution speed of the motors should be as high as possible ($>4.5$\,rad/s).
	\item The sampling frequenzy of the system should be higher than $130$\,Hz.
\end{itemize}

In order to improve the stabilization of the system the 2D-modelling can be replaced by a more exact 3D-modelling. Additionally the Ball's odometry can be included in the design of the controller.

\textbf{Keywords:} Ballbot, Omnidirectional Wheels, Gazebo, Matlab, Arduino 
\selectlanguage{ngerman} 