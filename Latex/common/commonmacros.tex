% Dieses File dient zum definieren n�tzlicher Befehle.
% Es soll lediglich als Beispiel dienen, wie Befehle definiert werden, und welche Befehle n�tzlich sein k�nnen
%


% Inhalt
% ======
%	Allgemeine Abk�rzungen
%	Makros f�r Referenzen (Abbildungen, Zitate, ...)
%	Makros f�r Abbildungen
%	Makros f�r Einheiten, Exponenten
%	Makros f�r Formeln
%	Makros f�r Entwurf
%   Definitionen f�r Umgebungen



% Allgemeine Abk�rzungen
% ======================
	\newcommand{\bzw}{bzw.\@\xspace}
	\newcommand{\Bzw}{Bzw.\@\xspace}
	\newcommand{\bzgl}{bzgl.\@\xspace}
	\newcommand{\ca}{ca.\@\xspace}
	\newcommand{\evtl}{evtl.\@\xspace}
	\newcommand{\usw}{usw.\@\xspace}
	\newcommand{\etc}{etc.\@\xspace}
	\newcommand{\vgl}{vgl.\@\xspace}
	\newcommand{\Vgl}{Vgl.\@\xspace}
	\newcommand{\bspw}{bspw.\@\xspace}
	\newcommand{\Bspw}{Bspw.\@\xspace}
	\newcommand{\ggf}{ggf.\@\xspace}
	\newcommand{\Ggf}{Ggf.\@\xspace}



	\newcommand{\dah}{d.\thinspace{}h.\@\xspace}
	\newcommand{\Dah}{D.\thinspace{}h.\@\xspace}
	\newcommand{\iA}{i.\thinspace{}A.\@\xspace}
	\newcommand{\IA}{i.\thinspace{}A.\@\xspace}
	\newcommand{\ua}{u.\thinspace{}a.\@\xspace}
	\newcommand{\Ua}{U.\thinspace{}a.\@\xspace}
	\newcommand{\uU}{u.\thinspace{}U.\@\xspace}
	\newcommand{\UU}{u.\thinspace{}U.\@\xspace}
	\newcommand{\zB}{z.\thinspace{}B.\@\xspace}
	\newcommand{\ZB}{Zum Beispiel\xspace}
	\newcommand{\zT}{z.\thinspace{}T.\@\xspace}
	\newcommand{\ZT}{Z.\thinspace{}T.\@\xspace}



% Makros f�r Referenzen (Abbildungen, Zitate, ...)
% ================================================

	% Referenzierung auf Abbildungen, Tabellen, etc. (Hyperref-f�hig)
	\newcommand{\figref}[1]{\hyperref[#1]{\figurename\ \ref*{#1}}}
	\newcommand{\tabref}[1]{\hyperref[#1]{\tablename\ \ref*{#1}}}
	\newcommand{\equref}[1]{\hyperref[#1]{Gl.~(\ref*{#1})}}
	\newcommand{\defref}[1]{\hyperref[#1]{Definition~\ref*{#1}}}
	\newcommand{\figvref}[1]{\hyperref[#1]{\figurename}\vref{#1}}
	\newcommand{\tabvref}[1]{\hyperref[#1]{\tablename}\vref{#1}}
	\newcommand{\equvref}[1]{\hyperref[#1]{Gl.~(\ref*{#1}) auf Seite \pageref*{#1}}}
	\newcommand{\pagerefh}[1]{\hyperref[#1]{Seite~\pageref*{#1}}}
	\newcommand{\secref}[1]{\hyperref[#1]{Abschnitt~\ref*{#1}}}
	\newcommand{\charef}[1]{\hyperref[#1]{Kapitel~\ref*{#1}}}
	\newcommand{\lstref}[1]{\hyperref[#1]{Listing~\ref*{#1}}}


	% Zitate mit Seitenangabe in Fu�note
%	\newcommand{\citep}[2]{\cite{#1}\footnote{Seite #2}}
%	\newcommand{\citepp}[2]{\cite{#1}\footnote{Seiten #2}}
	\newcommand{\citep}[2]{\cite{#1} (S. #2)}
	\newcommand{\citepp}[2]{\cite{#1} (S. #2)}
	
	
% Makros f�r Abbildungen
% ======================
	% zum Skalieren nach Ersetzen durch psfrag
	\newcommand{\incgraphicsw}[2]{\resizebox{#1}{!}{\includegraphics{#2}}}


% Textbausteine
% =============
	% Produktnamen
	\newcommand{\Matlab}{\textsc{Matlab}}
	\newcommand{\Matlabreg}{\textsc{Matlab}\textsuperscript{\tiny \textregistered}}
	\newcommand{\MatSim}{\textsc{Matlab/Simulink}}
	\newcommand{\Simulink}{\textsc{Simulink}}
	\newcommand{\Simulinkreg}{\textsc{Simulink}\textsuperscript{\tiny \textregistered}}



% Makros f�r Einheiten, Exponenten
% ================================

	\newcommand{\unit}[1] { \ensuremath{\mathrm{#1}}}
	
	% Wert mit Einheit (mit kleinem Leerzeichen dazwischen), aus Text- UND Math-Modus
	\newcommand{\valunit}[2]{\ensuremath{#1\,\mrm{#2}}}


	% "�C", im Text- oder Mathe-Modus
	\newcommand{\degC}{
		\ifmmode
			^\circ \mrm{C}%
		\else
			\textdegree C%
		\fi}

	\newcommand{\degree}{
		\ifmmode
			^\circ%
		\else
			\textdegree%
		\fi}
	
	% F�r Exponentenschreibweise ( Anwendung: 123\E{3} )
	\newcommand{\E}[1]{ \ensuremath{\cdot 10^{#1}} }
	
	\newcommand{\eexp}[1]{ \mathrm{e}^{#1} }
	\newcommand{\iu}{ \mathrm{j} }

	\newcommand{\todots}{ ,\,\hdots,\, }


% Makros f�r Formeln
% ==================

    \newcommand{\mat}[1]{{\ensuremath{\boldsymbol{\mathrm{#1}}}}}

	\newcommand{\AP} { \mathrm{AP} }
	\newcommand{\doti} {(i)^\cdot}

	% Definition f�r Vektor und Matizen
	\newcommand{\ve}[1]{\ensuremath{\mathbf{#1}}}
	\newcommand{\ma}[1]{\ensuremath{\mathbf{#1}}}

	% Definition f�r Vektor und Matizen
	\newcommand{\ves}[1]{\ensuremath{\boldsymbol{#1}}}
	\newcommand{\mas}[1]{\ensuremath{\boldsymbol{#1}}}
	
	
	\newcommand{\inprod}[2]{\langle #1,\,#2 \rangle}
	
	\newcommand{\ul}[1]{\underline{#1}}

	% gerades "d" (z.B. f�r Integral)
	\newcommand{\ud} { \mathrm{d} }
	
	% normaler Text in Formeln
	\newcommand{\tn}[1] { \textnormal{#1} }
	
	% nicht-kursive Schrift in Formeln
	\newcommand{\mrm}[1] { \mathrm{#1}}
	
	% gerades "T" f�r Transponiert
	\newcommand{\transp}{\mathrm{T}}
	
	% gerades "rg"
	\newcommand{\rang}[1]{\mathrm{rg}(#1)}

	% F�r geklammerte Ausdr�cke mit Index (Subscript)
	% (einmal mit kursiven Index, einmal mit geradem Index)
	\newcommand{\grpsb}[2] { \left( #1 \right)_{#2} }
	\newcommand{\grprsb}[2] { \left( #1 \right)_{\mathrm{#2}} }

	% Ableitungen und Integrale
		% "normale" Ableitung (mit geraden "d"s)
		\newcommand{\normd}[2] { \frac{\mathrm{d} #1 }{\mathrm{d} #2 } }
		\newcommand{\normdat}[3] { \left. \frac{\mathrm{d} #1 }{\mathrm{d} #2 } \right|_{#3} }
	
		% Materielle Ableitung
		\newcommand{\matd}[2] { \frac{\mathrm{D} #1 }{\mathrm{D} #2 } }
		\newcommand{\matdat}[3] { \left. \frac{\mathrm{D} #1 }{\mathrm{D} #2 } \right|_{#3} }
	
		% Partielle Ableitung
		\newcommand{\partiald}[2] { \frac{\partial #1 }{\partial #2 } }
		\newcommand{\partialdat}[3] { \left. \frac{\partial #1 }{\partial #2 } \right|_{#3} }
	
	
	% Transformationen
	\newcommand{\FT}[1] { \mathfrak{F} \left\{ #1 \right\} }
	\newcommand{\FTabs}[1]{\left| \mathfrak{F} \left\{ #1 \right\} \right|}
	\newcommand{\IFT}[1] { \mathfrak{F}^{-1} \left\{ #1 \right\} }
	\newcommand{\DFT}[1]{\mathrm{DFT} \left\{ #1 \right\}}
	\newcommand{\DFTabs}[1]{\left| \mathrm{DFT} \left\{ #1 \right\} \right|}

	\newcommand{\Laplace}[1]{\mathfrak{L}\left (#1\right)} % L-Transformation
	\newcommand{\InvLaplace}[1]{\mathfrak{L^{-1}}\left (#1\right )} % L-Transformation, invers
	\newcommand{\invtrans}{\quad\bullet\!\!-\!\!\!-\!\!\circ\quad}
	\newcommand{\trans}{\quad\circ\!\!-\!\!\!-\!\!\bullet\quad}


	\newcommand{\mlfct}[1]{{\tt #1}}
	\newcommand{\mlvar}[1]{{\tt #1}}


	% Manche textcomp-Zeichen funktionieren mit dem TU-Design nicht, diese k�nnen dann mit diesem
	% Befehl gesetzt werden.
	\newcommand{\textcompstdfont}[1]{{\fontfamily{cmr} \fontseries{m} \fontshape{n} \selectfont #1}}
	


% =================================================================================
% Defintionen f�r Mathe-Umgebungen
% =================================================================================
	
\newtheorem{theorem}{Satz}
\newtheorem{lemma}[theorem]{Lemma}	% Selber Z�hler wie theorem
\newtheorem{definition}{Definition}
% =================================================================================


% =================================================================================
% Defintionen f�r Beispiel-Umgebung
% =================================================================================
\newcounter{examplenumber}[chapter]      % Neuer Counter bspnummer nummeriert nach Kapitel
\def\theexamplenumber{\thechapter.\arabic{examplenumber}}

\newenvironment{example}[1][]
{\vskip 3\parskip plus 1pt minus 1pt \refstepcounter{examplenumber}
\vspace{.3cm} \begin{addmargin}[1cm]{0cm} \noindent \textbf{Beispiel \theexamplenumber}: \textbf{#1}\par}
{\end{addmargin} \par \vspace{.3cm}}

% Alternative, einfachere Beispielumgebung:
% \newtheorem{example}{Beispiel}
% =================================================================================




% =================================================================================
% Definitionen f�r Listingsumgebung
% =================================================================================

\lstloadlanguages{Matlab}

\lstdefinestyle{Matlab_colored_smallfont}
{
	language = Matlab,
	tabsize = 4,
	framesep = 3mm,
	frame=tb,
	classoffset = 0,	
	basicstyle = \footnotesize\ttfamily,
	keywordstyle = \bfseries\color[rgb]{0,0,1},
	commentstyle = \itshape\color[rgb]{0.133,0.545,0.133},
	stringstyle = \color[rgb]{0.627,0.126,0.941},
	extendedchars = true,
	breaklines = true,
	prebreak = \textrightarrow,
	postbreak = \textleftarrow,
	%escapeinside = {(*@}{@*)},
	%moredelim = [s][\itshape\color[rgb]{0.5,0.5,0.5}]{[.}{.]},
	numbers = left,
	numberstyle = \tiny,
	stepnumber = 5
}

\lstdefinestyle{Matlab_colored}
{
	language = Matlab,
	tabsize = 4,
	framesep = 3mm,
	frame=tb,
	classoffset = 0,	
	basicstyle = \ttfamily,
	keywordstyle = \bfseries\color[rgb]{0,0,1},
	commentstyle = \itshape\color[rgb]{0.133,0.545,0.133},
	stringstyle = \color[rgb]{0.627,0.126,0.941},
	extendedchars = true,
	breaklines = true,
	prebreak = \textrightarrow,
	postbreak = \textleftarrow,
	%escapeinside = {(*@}{@*)},
	%moredelim = [s][\itshape\color[rgb]{0.5,0.5,0.5}]{[.}{.]},
	numbers = left,
	numberstyle = \tiny,
	stepnumber = 5
}


\lstdefinestyle{C_colored_smallfont}
{
	language=C,
	tabsize = 4,
	framesep = 3mm,
	frame=tb,	
	classoffset = 0,	
	basicstyle = \footnotesize\ttfamily,
	keywordstyle = \bfseries\color[rgb]{0,0,1},
	commentstyle = \itshape\color[rgb]{0.133,0.545,0.133},
	stringstyle = \color[rgb]{0.627,0.126,0.941},
	extendedchars = true,
	breaklines = true,
	prebreak = \textrightarrow,
	postbreak = \textleftarrow,
	%escapeinside = {(*@}{@*)},
	%moredelim = [s][\itshape\color[rgb]{0.5,0.5,0.5}]{[.}{.]},
	numbers = left,
	numberstyle = \tiny,
	stepnumber = 5
}

\lstdefinestyle{C_colored}
{
	language=C,
	tabsize = 4,
	framesep = 3mm,
	frame=tb,
	classoffset = 0,	
	basicstyle = \ttfamily,
	keywordstyle = \bfseries\color[rgb]{0,0,1},
	commentstyle = \itshape\color[rgb]{0.133,0.545,0.133},
	stringstyle = \color[rgb]{0.627,0.126,0.941},
	extendedchars = true,
	breaklines = true,
	prebreak = \textrightarrow,
	postbreak = \textleftarrow,
	%escapeinside = {(*@}{@*)},
	%moredelim = [s][\itshape\color[rgb]{0.5,0.5,0.5}]{[.}{.]},
	numbers = left,
	numberstyle = \tiny,
	stepnumber = 5
}
