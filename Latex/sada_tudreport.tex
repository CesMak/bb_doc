% =================================================================================
% Hier ausw�hlen, ob TUD-Design oder nicht
% =================================================================================
\newif\ifTUDdesign
\TUDdesigntrue					% TUD-Design
%\TUDdesignfalse				% F�r Rechner ohne installierte TUDdesign-Pakete
% =================================================================================


% =================================================================================
% Hier Daten f�r studentische Arbeit eingeben
% =================================================================================
\newcommand{\SADATyp}{Projektseminar}
\newcommand{\SADATitel}{Aufbau und Regelung eines Ballbots}
\newcommand{\SADAStadt}{Darmstadt}
\newcommand{\SADAAutor}{Florian M�ller \newline Markus Lamprecht \newline Michael Suffel}
\newcommand{\SADABetreuer}{Dr.-Ing. Eric Lenz}
\newcommand{\SADABetreuerII}{}
\newcommand{\SADABetreuerIII}{}
\newcommand{\SADABegin}{16. Oktober 2017}
\newcommand{\SADAAbgabe}{16. Februar 2018}
\newcommand{\SADASeminar}{16. Februar 2018}
% =================================================================================


% =================================================================================
% Auswahl des IAT-Fachgebiets (rtm / rtp)
% =================================================================================
\newif\ifrtm
\rtmtrue	% rtm
%\rtmfalse	% rtp
% =================================================================================


% =================================================================================
% Erkl�rung, dass die Arbeit ohne Hilfe Dritter etc. erstellt wurde
% =================================================================================
\def\SADAVarianteErklaerung{ETIT}		% FB 18, Elektrotechnik
%\def\SADAVarianteErklaerung{MBDA}		% FB 16, Maschinenbau, Diplomarbeit
%\def\SADAVarianteErklaerung{MBSA}		% FB 16, Maschinenbau, Studienarbeit
% =================================================================================


% =================================================================================
% Ausnahmen von der automatischen Silbentrennung
% =================================================================================
\hyphenation{Aktu-ali-sie-rung Screen-shots Pa-rallel-ro-bo-ter Zu-stands-raum-mo-del-le nach-voll-zieh-bar Pro-jekt-se-mi-nar}
% =================================================================================


% =================================================================================
% Hier NIICHTS �ndern!
% =================================================================================
\ifTUDdesign
	\documentclass[11pt, twoside, colorback, accentcolor=tud2c, nopartpage, bigchapter, fleqn, ngerman, longdoc]{tudreport}
\else
	\documentclass[11pt, a4paper, twoside, fleqn, ngerman]{scrreprt}
  % F�r Entwurf auf Rechnern ohne installierte TUDdesign-Pakete	
	\usepackage{exscale}	% Korrektur math-Zeichen
	\usepackage{eurosym}
\fi
% Dieses File dient zum einbinden wichtiger und n�tzlicher Pakete.
% Nicht alle Pakete m�ssen verwendet werden.
%
\usepackage{t1enc}			% evtl. dc-Fonts
\usepackage[T1]{fontenc}	% F�r Silbentrennung bei W�rten mit Sonderzeichen (z.B. Umlaute)
\usepackage[latin1]{inputenc}
							% Um Sonderzeichen (�, �, �, ...) direkt eingeben zu k�nnen
\usepackage[english,ngerman]{babel}
							% F�r Sprachenspezifisches
							% ngerman ist schon als globale Option definiert

\usepackage{helvet}			% Helvetica als Standard-Sans-Schriftart
\usepackage[stable]{footmisc}
\usepackage{booktabs}


\usepackage{graphicx}		% zum Einbinden von Postscript
\usepackage{psfrag}			% Beschriftung der Bilder
\usepackage{amsmath}		% Mehr mathematischen Formelsatz
%\usepackage{amssymb}		% Mehr mathematische Symbole
%\usepackage{amsthm}

\usepackage{float}			% F�r Parameter [H] bei Flie�objekten

\usepackage{epsfig}			% Um eps-Bilder einzubinden
\usepackage{scrhack}    % Um Warnung "float@addtolists detected" zu unterdr�cken 
\usepackage{subfig}			% F�r Unterabbildungen
\usepackage{ltxtable} 		% Vereinigt TabularX und Longtable
\usepackage{longtable}
\usepackage{rotating}		% Zum Drehen von Objekten
\usepackage{bibgerm}		% F�r deutsche Literaturverwaltung
%\usepackage{wrapfig}		% F�r kleine Bilder am Rand
%\usepackage{floatflt}		% Alternative zu wrapfig
%\usepackage[hang]{caption}	% Damit mehrzeilige Bildunterschriften gut aussehen
\usepackage{upgreek}		% F�r nicht-kursive kleine griechischen Buchstaben

\usepackage{multirow}		% F�r mehrzeilige Felder in Tabellen

\usepackage{textcomp}		% F�r Sonderzeichen im normalen Text
							% (offensichtlich in tudreport schon eingebunden)
\usepackage[ngerman]{varioref}		% F�r vref
\usepackage{color}			% F�r farbigen Text
\usepackage{placeins}		% F�r \FloatBarrier
\usepackage{xspace}
\usepackage{icomma}			% Damit nach Dezimalkommas kein Abstand eingef�gt wird
							% (in math-Umgebungen)

\usepackage{cancel}			% Zum Wegstreichen von Gleichungstermen

\usepackage{array}			% F�r Zellentyp "m{}" in tabular-Umgebungen (Vertikal zentriert)

\usepackage{listings}		% Um formatierten Quellcode einzubinden
\usepackage{moreverb}		% F�r Umgebung "`verbatimtab"' (Verbatim mit Tabs)
\renewcommand{\verbatimtabsize}{4\relax}	% Standardtabweite in "`verbatimtab"'

											% ist 4 Zeichen


% Das Packet hyperref immer als letztes einbinden (nur bookmark darf danach kommen)!
%\usepackage[ps2pdf, colorlinks=false, pdfborder={0 0 0}]{hyperref}
%\usepackage[ps2pdf]{hyperref}	% F�r Verlinkungen im erzeugten pdf
\usepackage{hyperref}	% F�r Verlinkungen im erzeugten pdf
\usepackage{bookmark}
\usepackage{url}

% Markus includes:
\usepackage{array}
\newcolumntype{L}[1]{>{\raggedright\let\newline\\\arraybackslash\hspace{0pt}}m{#1}}
\newcolumntype{C}[1]{>{\centering\let\newline\\\arraybackslash\hspace{0pt}}m{#1}}
\newcolumntype{R}[1]{>{\raggedleft\let\newline\\\arraybackslash\hspace{0pt}}m{#1}}
\usepackage{footnote}
\makesavenoteenv{tabular} % footnotes in tabular env.

% Florian includes:
\usepackage{tikz}
%\usepackage{etoolbox}
%\makeatletter
%\AtBeginDocument{
%\patchcmd{\use@@tikzlibrary}{\global}{}{}{}
%}
%\makeatother
\usepackage{pgfplots} 
\usepackage{pgfgantt}
\usepackage{pdflscape}
\pgfplotsset{compat=newest} 
\pgfplotsset{plot coordinates/math parser=false}
\newlength\figH
\newlength\figW
\setlength{\figH}{7cm}
\setlength{\figW}{14cm}				% verwendete Pakete einbinden
\input{common/setup.tex}					% LaTeX-Einstellungen
\input{common/commonmacros.tex}		% oft verwendete Befehle
% =================================================================================


% =================================================================================
% Hier beginnt das eigentliche Dokument
% =================================================================================
\begin{document}
\input{common/preface.tex} % Titelseite, Aufgabenstellung, Erkl�rung, Abstract, Inhaltsverzeichnis, etc.

\chapter{Einleitung}
4 und 3 ballbots


\chapter{Modellbildung und Regelung}

\section{Model}
F�r die Systembeschreibung wird der dreidimensionale Ballbot in drei unabh�ngige planare Modelle aufgeteilt. In jeder Ebene wird das System vereinfacht als eine Kugel, ein virtuelles Rad und einen K�rper angenommen und besitzt somit zwei Freiheitsgrade, die sich in eine Translation bzw. Rotation des Balles und eine Rotation des K�rpers aufteilen lassen.\cite{Manual} 
\chapter{Konstruktion} \label{ch:konstruktion}

\section{Konzeptionierung} \label{sec:konzeptionierung}

\begin{figure}[H]%
	\centering
	{\includegraphics[scale=6.4]{Bilder/Michael/TUD_Ballbot_V30_MS_mit_Raeder_BAll.jpg} }
	\caption{Oberbau, bestehend aus vier Etagen}
	\label{fig:oberbau}
\end{figure}

F�r den Entwurf eines auf einem Ball balancierenden Roboters kann das System zur Vereinfachung als ein inverses Pendel betrachtet werden, dessen Drehachse auf einer beweglichen Plattform gelagert ist. Dies ist in Abbildung \ref{fig:invPendulum} beispielhaft dargestellt. Unter dieser Annahme k�nnen erste Absch�tzungen �ber das dynamische Verhalten getroffen werden. Aus der Anschauung heraus kann direkt gefolgert werden, dass der Pendelarm mit der L�nge $l$ einen Einfluss auf die Pendeldynamik des Systems haben muss. Leitet man nun wie in \cite{pendel}, unter Verwendung der Momentenbilanz das nichtlineare Differenzialgleichungssstem zweiter Ordnung des Auslenkwinkels $\vartheta$ sowie der Schlittenstrecke $x_w$ her, so erh�lt man die Gleichungen: 

        \begin{align} \label{eq:InvPendelThetaDotDot}
            \ddot{\vartheta} &= \frac{l\cdot m}{I}\cdot [-\cos{\vartheta} \cdot \ddot{x}_{w}+g\cdot  \sin{\vartheta}] ,\\   
            \label{eq:InvPendelXDotDot}
            \ddot{x}_w &= \frac{F_{motor}-F_{Reib}}{M+m} + \frac{m\cdot l}{M+m}\dot{\vartheta}^2 \cdot \sin{\vartheta} - \ddot{\vartheta} . 
        \end{align}


Setzt man nun das Massentr�gheitsmoment des inversen Pendels

        \begin{align} \label{eq:tr�gheitsmoment}
            I &= \frac{4}{3}m\cdot l^2 ,   
        \end{align}
in Gl.\,\ref{eq:InvPendelThetaDotDot} ein, so ergibt sich:  

        \begin{align} \label{eq:InvPendelThetaDotDot_Neu}
            \ddot{\vartheta} &= \frac{3}{4\cdot l}\cdot[-\cos{\vartheta} \cdot \ddot{x}_w+g \cdot  \sin{\vartheta}] . 
        \end{align}

\begin{figure}[H]%
	\centering
	{ \scalebox{1}{{\input{./Bilder/Michael/Schlitten.pdf_tex}}}}%
	\caption{Darstellung eines Ballbots als inverses Pendel}%
	\label{fig:invPendulum}%
\end{figure}%\newline

Anhand der Gleichung (\ref{eq:InvPendelThetaDotDot_Neu}) l�sst sich wie bereits vermutet eine Abh"angikeit der Winkelbeschleunigung $\ddot{\vartheta}$ von der Hebelarml"ange $l$ erkennen. Da sich die Hebelarml�nge jedoch aus der Distanz zwischen Drehachse und Massenschwerpunkt zusammensetzt, bestimmt die Verteilung der Ballbotmasse die Hebelarml�nge des inversen Pendels. Durch das Vergr��ern der Hebelarml�nge $l$ kann so die Winkelbeschleunigung $\ddot{\vartheta}$ f�r kleine Winkelauslenkungen $\Delta \vartheta$ verringert werden. Dadurch ist das System leichter zu stabilisieren. Bei Betrachtung der Gleichung (\ref{eq:InvPendelXDotDot}) l�sst sich jedoch erkennen, dass mit einer Verl�ngerung des Hebelarms $l$ eine gr��ere Beschleunigung $\ddot{x}_w$  des Schlittens einhergehen muss, die durch entsprechend starke Drehmomente realisiert werden kann.

Wendet man dieses Prinzip eines gro�en Hebelarms auf den Ballbot an, so stellt sich jedoch heraus, dass es nur bedingt anwendbar ist. Ursache hierf�r ist die Schnittstelle zwischen omnidirektionalen R�dern und Ball, da hier aufgrund des Reibkoeffizienten nur ein begrenztes Drehmoment �bertragen werden kann. Der Reibkoeffizient begrenzt somit die gew�nschte Sollbeschleunigung des Balles. 

Da eine optimale Auslegung der Pendelarml�nge rechnerisch nur schwer zu bestimmen ist, w�re es w�nschenswert, die Hebelarml�nge durch eine flexible Hardware ver�ndern zu k�nnen. Durch das Hinzuf�gen bzw. Entfernen von weiteren Ebenen im Aufbau des Ballbots kann der Massenschwerpunkt und somit die Pendelarml�nge des Ballbots variabel eingestellt werden. So kann die optimale Konfiguration sehr einfach experimentell bestimmt werden. 

Als Hardware f�r den Ballbot hat man sich daher f�r das TurtleBot3-Paket in der Burger-Variante entschieden. Auf diese soll im Kapitel \ref{sec:Hardware} tiefer eingegangen werden. Bei der Wahl des Balles sowie der Antriebsanordnung hat man sich die Konzepte bereits funktionierender Ballbots als Grundlage hergenommen. Die folgenden Kapitel gehen dabei n�her auf die einzelnen Konstruktionsschritte ein. 

\section{Verwendete Hardware} \label{sec:Hardware}
 Beim Entwurf des Ballbots wurde, wie bereits erw�hnt, auf die leistungsf"ahige Hardware eines TurtleBot3 in der \glqq Burger\grqq-Variante zur"uck gegriffen. Dieses Paket umfasst neben einem robusten mechanischen Aufbau auch alle grundlegenden elektronischen Komponenten wie Sensoren, Mikrocontrollern und Antriebseinheiten, die f�r die Entwicklung eines auf einem Ball balancierenden Roboters ben�tigt werden. Die wichtigsten Komponenten werden in den nachfolgenden Unterkapitel n�her erl�utert.

\subsection{OpenCR-Board} \label{sec:openCr}

Das im Projekt verwendete OpenCR-Board (in der Version 1.0), wie in Abbildung \ref{fig:opencr_board} dargestellt, ist ein Mikrocontroller, der f�r Roboterprojekte in der Gr��e des TurtleBot3-Pakets perfekt geeignet ist. Das gesamte Board, von der Hardware bis zur Software, ist durch Open-Source-Lizenzen frei verf�gbar. Es bietet mit insgesamt 10 verschiedenen Kommunikationsstellen und einer integrierten inertialen Messeinheit (IMU)\footnote{Die verwendete IMU ist die MPU-9250 siehe \url{https://www.invensense.com/products/motion-tracking/9-axis/mpu-9250/}.} bei der Entwicklung von Robotersystemen. Dar�ber hinaus ist das OpenCR-Board mit einem leistungsstarken ARM Cortex-M7 Prozessor ausgestattet.

\begin{figure}[H]%H !htbp
	\centering
	{\includegraphics[scale=0.45]{Bilder/Michael/opencr.png} }
	\caption{�bersicht der Schnittstellen des OpenCR-Boards \cite{openCR}}
	\label{fig:opencr_board}
\end{figure}

%https://github.com/ROBOTIS-GIT/emanual/blob/master/docs/en/platform/turtlebot3/appendix_opencr1_0.md

\subsection{Sensorik} \label{sec:sensoren}
Am Ballbot wurden mehrere Sensoren verbaut, die f�r verschiedene Aufgabenbereiche ben�tigt werden. Tabelle \ref{tab:sensoren} zeigt dabei eine �bersicht der Sensoren, die zur Stabilisierung bzw. zur Lokalisation verwendet werden.  


\begin{table}[H]
	\caption["Ubersicht Sensoren]{Sensoren, die im Ballbot verbaut wurden} \label{tab:sensoren}
  		\begin{center}				
		\begin{longtable}{|c||c|c|c|}
		\hline 
  		\multicolumn{4}{|c|}{�bersicht Sensoren} \\ 
  		\hline
  		Kategorie & Stabilisierung & \multicolumn{2}{c|}{Lokalisierung/Interaktion} \\ 
  		\hline
  		\hline
  		Bezeichnung & IMU & Kamera & Laser Distance Sensor\\
  		& MPU-9250 & Intel Realsense R200 & LDS-01\\
  		\hline	\hline
  		
  		 & & & \\
         Taktfrequenz: & 4Hz - 8kHz & 2.5GHz & 1,8 kHz\\
         & & & \\
         Features: & Gyroskop & IR Laser Projector System & DA: 120mm - 3500mm\\
                  & Accelerometer & Full HD RGB Color Stream & Scan Rate: 300$\pm$10 rpm\\
                  & Magnetometer & Onboard Imaging ASIC & Angular Resolution: $1^{\circ}$\\	
  		\hline
		\end{longtable}
  		\end{center}
\end{table} 


\subsection{Motoren} \label{sec:Motoren}

\begin{figure}[H]%
	\centering
	{\includegraphics[scale=6.4]{Bilder/Michael/Motoren.jpg} }
	\caption{Motorenspezifikationen}
	\label{fig:motorenspezifikationen}
\end{figure}


Im TurtleBot3-Paket enthalten sind weiterhin auch die Motoren, welche f"ur den Ballbotaufbau verwendet werden. Es handelt sich dabei um Dynamixel Servomotoren der Baureihe XM430-W350-T. Die Motoren haben sich als sehr robust erwiesen und k"onnen durch eine hohe "Ubersetzung von $i=1:353.5$ ein variables ausgangsseitiges Drehmoment von bis zu $3.8$\,Nm bei $11.1$\,V \cite{XM430} erzeugen. Dies ist f"ur die gegebenen Systemkonfiguration mehr als ausreichend. Weiterhin werden von Dynamixel bereits verschiedene Betriebsarten bereitgestellt, in denen die Motoren betrieben werden k"onnen. So bieten die Motoren neben einer Positionsregelung, einer Geschwindigkeitsregelung auch eine Stromregelung. Durch die lineare Abh"angikeit zwischen Drehmoment und Strom kann somit das abtriebseitige Drehmoment geregelt werden.
 
Vom OpenCR-Board werden mittels dem RS485-Protokoll sogennante Units �bertragen. Die Units werden dann intern im Motor in einen realen Sollstrom umgewandelt. Der Strom kann �ber die Formel 

\begin{align} \label{eq:CurrentUnit}
I &= k_{unit}\cdot Units, && \text{mit}  & k_{unit} &= 2,69\cdot 10^{-3} \frac{A}{Unit}
\end{align}

berechnet werden. Anschlie�end kann mit Hilfe der Drehmoment(Nm)-Strom(A)-Kennlinie aus dem Datenblatt \cite{XM430} die spezifische Motorkonstante zu $k_{motor} = 1,63$\,Nm/A bestimmt werden.  
  bestimmt.  Dadurch kann das vom Motor erzeugte Drehmoment aus den Units mittels der Formel 
  \begin{align} \label{eq:CurrentTorque}
  M &= k_{motor} \cdot I \\
    &= k_{motor}\cdot k_{unit} \cdot Units
  \end{align}
  berechnet werden. Insgesamt ergibt sich folgender Gesamtumrechnungsfaktor 
    \begin{align} \label{eq:CurrentTorque}
    k &= k_{motor}\cdot k_{unit} = 228\cdot \frac{Nm}{Unit} 
    \end{align}
    
In der technischen Umsetzung hat sich herausgestellt, dass die Motoren zueinander ein unterschiedliches Verhalten aufweisen und somit die Konstante $k$ nicht f�r alle drei Motoren verwendet werden kann. Daher wurde die Konstante $k$ experimentell bestimmt.
Hierzu wird ein Pr�fhebel der L�nge $l_{pruef}$ f�r die Motoren konstruiert, der w�hrend des Betriebs der Motoren auf eine Waage dr�ckt. Die Motoren werden mit einer bestimmten Folge von dimensionslosen Einheiten angesteuert und jeweils das angezeigte Gewicht der Waage $m$ notiert.
Mit der Beziehung
\begin{align} \label{eq:FMG}
F &= m \cdot g \nonumber
\end{align}
eingesetzt in
\begin{align} 
 M &= F \cdot l_{\text{pruef}}
\end{align}
kann das resultierende Drehmoment $M$ berechnet werden und somit die Konstante $k$ f�r jeden einzelnen Motor.

Bei den Messungen wurde festgestellt, dass das von den Motoren erzeugte Drehmoment einen Drift\footnote{Unter einem Drift ist hier gemeint, dass bei konstantem Anlegen von Units (Stromwert) das Drehmoment �ber die Zeit abnimmt.} aufweist. Dieser Drift konnte reduziert werden, in dem man statt einmaligen Drehmomentbefehlen, die Befehle in Form einer Pulsung wiederkehrend an den Motor �bergeben hat. Die neuen $k$-Faktoren sind �ber die Ergebnisse der Messungen, dargestellt in den Abbildungen \ref{fig:Motorkonstante1}, \ref{fig:Motorkonstante2} und \ref{fig:Motorkonstante3}, mittels der Methode der kleinsten Quadrate berechnet worden. Aus diesen Abbildungen ist zu erkennen, dass sich die Drehmoment-Unit-Konstanten f�r die einzelnen baugleichen Motoren unterscheiden. Dieses Verhalten wurde bei der Implementierung in Kapitel \ref{ch:Implementierung} ber�cksichtigt.
\begin{figure}[!htbp]%
	\centering
	{\includegraphics[scale=0.55]{Bilder/Michael/Motor1.png} }
	\caption{Bestimmung der Drehmoment-Unit-Konstante f�r Motor 1}
	\label{fig:Motorkonstante1}
\end{figure}

\begin{figure}[!htbp]%
	\centering
	{\includegraphics[scale=0.55]{Bilder/Michael/Motor2.png} }
	\caption{Bestimmung der Drehmoment-Unit-Konstante f�r Motor 2}
	\label{fig:Motorkonstante2}
\end{figure}

\begin{figure}[!htbp]%
	\centering
	{\includegraphics[scale=0.55]{Bilder/Michael/Motor3.png} }
	\caption{Bestimmung der Drehmoment-Unit-Konstante f�r Motor 3}
	\label{fig:Motorkonstante3}
\end{figure}      
            
\section{Ballbot-Design} \label{sec:entwurf}
Die mechanische Systemkonfiguration kann in drei Teile untergliedert werden. Dies ist zum einen der Oberbau (Kap.\ref{sec:oberbau}), der Unterbau (Kap.\ref{sec:unterbau}) sowie der Ball (Kap.\ref{sec:ball}), auf dem das Gesamtsystem bestehend aus Ober- und Unterbau balancieren wird. Nachfolgende Kapitel dokumentieren die Entwicklungsprozesse der einzelnen Teile. 

\subsection{Oberbau} \label{sec:oberbau}

\begin{figure}[H]%
	\centering
	{\includegraphics[scale=6.4]{Bilder/Michael/Oberbau.jpg} }
	\caption{Oberbau, bestehend aus vier Etagen}
	\label{fig:oberbau}
\end{figure}

Die Konfiguration des Oberbaus hat sich experimentell unter Ber�cksichtigung der in Kapitel \ref{ch:Modellierung} hergeleiteten Beziehungen ergeben. In der untersten Ebene ist die Batterie platziert. Dar�ber folgt eine Ebene mit einem Up-Level Computer, der f�r rechenintensiver Aufgaben wie der Lokalisation ben�tigt wird. Dieser wird jedoch in dieser Arbeit nicht weiter ber�cksichtigt. Anschlie�end folgt das OpenCR-Board mit den dar�ber platzierten Sensoren zur Lokalisierung des Roboters im Raum, vgl. Abbildung \ref{fig:oberbau}. 


\subsection{Unterbau} \label{sec:unterbau}

\begin{figure}[htb!]%
	\centering
	{\includegraphics[scale=6.4]{Bilder/Michael/Unterbau.jpg} }
	\caption{Unterbau}
	\label{fig:Unterbau}
\end{figure}

Die Aufgabe des Unterbaus besteht in der Positionierung der Antriebe. Dabei sollte dieser m�glichst leicht und zugleich stabil konstruiert werden. 

Damit das Drehmoment der Motoren optimal auf den Ball �bertragen werden kann, m�ssen die R�der gegen�ber der z-Achse\footnote{Die z-Achse ist die zum Ballbot vertikale Achse.} des Ballbots einen Winkel von $\alpha=45^{\circ}$ einschlie�en. Weiterhin wurden die R�der um einen Winkel von $\beta=120^{\circ}$ zueinander versetzt befestigt. Diese konstruktionsbedingten Winkel sind in Abbildung \ref{fig:konstruktions�bersicht} dargestellt.

\begin{figure}[H]%
	\centering
	{ \scalebox{1}{{\chapter{Konstruktion} \label{ch:konstruktion}

\section{Konzeptionierung} \label{sec:konzeptionierung}

\begin{figure}[H]%
	\centering
	{\includegraphics[scale=6.4]{Bilder/Michael/TUD_Ballbot_V30_MS_mit_Raeder_BAll.jpg} }
	\caption{Oberbau, bestehend aus vier Etagen}
	\label{fig:oberbau}
\end{figure}

F�r den Entwurf eines auf einem Ball balancierenden Roboters kann das System zur Vereinfachung als ein inverses Pendel betrachtet werden, dessen Drehachse auf einer beweglichen Plattform gelagert ist. Dies ist in Abbildung \ref{fig:invPendulum} beispielhaft dargestellt. Unter dieser Annahme k�nnen erste Absch�tzungen �ber das dynamische Verhalten getroffen werden. Aus der Anschauung heraus kann direkt gefolgert werden, dass der Pendelarm mit der L�nge $l$ einen Einfluss auf die Pendeldynamik des Systems haben muss. Leitet man nun wie in \cite{pendel}, unter Verwendung der Momentenbilanz das nichtlineare Differenzialgleichungssstem zweiter Ordnung des Auslenkwinkels $\vartheta$ sowie der Schlittenstrecke $x_w$ her, so erh�lt man die Gleichungen: 

        \begin{align} \label{eq:InvPendelThetaDotDot}
            \ddot{\vartheta} &= \frac{l\cdot m}{I}\cdot [-\cos{\vartheta} \cdot \ddot{x}_{w}+g\cdot  \sin{\vartheta}] ,\\   
            \label{eq:InvPendelXDotDot}
            \ddot{x}_w &= \frac{F_{motor}-F_{Reib}}{M+m} + \frac{m\cdot l}{M+m}\dot{\vartheta}^2 \cdot \sin{\vartheta} - \ddot{\vartheta} . 
        \end{align}


Setzt man nun das Massentr�gheitsmoment des inversen Pendels

        \begin{align} \label{eq:tr�gheitsmoment}
            I &= \frac{4}{3}m\cdot l^2 ,   
        \end{align}
in Gl.\,\ref{eq:InvPendelThetaDotDot} ein, so ergibt sich:  

        \begin{align} \label{eq:InvPendelThetaDotDot_Neu}
            \ddot{\vartheta} &= \frac{3}{4\cdot l}\cdot[-\cos{\vartheta} \cdot \ddot{x}_w+g \cdot  \sin{\vartheta}] . 
        \end{align}

\begin{figure}[H]%
	\centering
	{ \scalebox{1}{{\input{./Bilder/Michael/Schlitten.pdf_tex}}}}%
	\caption{Darstellung eines Ballbots als inverses Pendel}%
	\label{fig:invPendulum}%
\end{figure}%\newline

Anhand der Gleichung (\ref{eq:InvPendelThetaDotDot_Neu}) l�sst sich wie bereits vermutet eine Abh"angikeit der Winkelbeschleunigung $\ddot{\vartheta}$ von der Hebelarml"ange $l$ erkennen. Da sich die Hebelarml�nge jedoch aus der Distanz zwischen Drehachse und Massenschwerpunkt zusammensetzt, bestimmt die Verteilung der Ballbotmasse die Hebelarml�nge des inversen Pendels. Durch das Vergr��ern der Hebelarml�nge $l$ kann so die Winkelbeschleunigung $\ddot{\vartheta}$ f�r kleine Winkelauslenkungen $\Delta \vartheta$ verringert werden. Dadurch ist das System leichter zu stabilisieren. Bei Betrachtung der Gleichung (\ref{eq:InvPendelXDotDot}) l�sst sich jedoch erkennen, dass mit einer Verl�ngerung des Hebelarms $l$ eine gr��ere Beschleunigung $\ddot{x}_w$  des Schlittens einhergehen muss, die durch entsprechend starke Drehmomente realisiert werden kann.

Wendet man dieses Prinzip eines gro�en Hebelarms auf den Ballbot an, so stellt sich jedoch heraus, dass es nur bedingt anwendbar ist. Ursache hierf�r ist die Schnittstelle zwischen omnidirektionalen R�dern und Ball, da hier aufgrund des Reibkoeffizienten nur ein begrenztes Drehmoment �bertragen werden kann. Der Reibkoeffizient begrenzt somit die gew�nschte Sollbeschleunigung des Balles. 

Da eine optimale Auslegung der Pendelarml�nge rechnerisch nur schwer zu bestimmen ist, w�re es w�nschenswert, die Hebelarml�nge durch eine flexible Hardware ver�ndern zu k�nnen. Durch das Hinzuf�gen bzw. Entfernen von weiteren Ebenen im Aufbau des Ballbots kann der Massenschwerpunkt und somit die Pendelarml�nge des Ballbots variabel eingestellt werden. So kann die optimale Konfiguration sehr einfach experimentell bestimmt werden. 

Als Hardware f�r den Ballbot hat man sich daher f�r das TurtleBot3-Paket in der Burger-Variante entschieden. Auf diese soll im Kapitel \ref{sec:Hardware} tiefer eingegangen werden. Bei der Wahl des Balles sowie der Antriebsanordnung hat man sich die Konzepte bereits funktionierender Ballbots als Grundlage hergenommen. Die folgenden Kapitel gehen dabei n�her auf die einzelnen Konstruktionsschritte ein. 

\section{Verwendete Hardware} \label{sec:Hardware}
 Beim Entwurf des Ballbots wurde, wie bereits erw�hnt, auf die leistungsf"ahige Hardware eines TurtleBot3 in der \glqq Burger\grqq-Variante zur"uck gegriffen. Dieses Paket umfasst neben einem robusten mechanischen Aufbau auch alle grundlegenden elektronischen Komponenten wie Sensoren, Mikrocontrollern und Antriebseinheiten, die f�r die Entwicklung eines auf einem Ball balancierenden Roboters ben�tigt werden. Die wichtigsten Komponenten werden in den nachfolgenden Unterkapitel n�her erl�utert.

\subsection{OpenCR-Board} \label{sec:openCr}

Das im Projekt verwendete OpenCR-Board (in der Version 1.0), wie in Abbildung \ref{fig:opencr_board} dargestellt, ist ein Mikrocontroller, der f�r Roboterprojekte in der Gr��e des TurtleBot3-Pakets perfekt geeignet ist. Das gesamte Board, von der Hardware bis zur Software, ist durch Open-Source-Lizenzen frei verf�gbar. Es bietet mit insgesamt 10 verschiedenen Kommunikationsstellen und einer integrierten inertialen Messeinheit (IMU)\footnote{Die verwendete IMU ist die MPU-9250 siehe \url{https://www.invensense.com/products/motion-tracking/9-axis/mpu-9250/}.} bei der Entwicklung von Robotersystemen. Dar�ber hinaus ist das OpenCR-Board mit einem leistungsstarken ARM Cortex-M7 Prozessor ausgestattet.

\begin{figure}[H]%H !htbp
	\centering
	{\includegraphics[scale=0.45]{Bilder/Michael/opencr.png} }
	\caption{�bersicht der Schnittstellen des OpenCR-Boards \cite{openCR}}
	\label{fig:opencr_board}
\end{figure}

%https://github.com/ROBOTIS-GIT/emanual/blob/master/docs/en/platform/turtlebot3/appendix_opencr1_0.md

\subsection{Sensorik} \label{sec:sensoren}
Am Ballbot wurden mehrere Sensoren verbaut, die f�r verschiedene Aufgabenbereiche ben�tigt werden. Tabelle \ref{tab:sensoren} zeigt dabei eine �bersicht der Sensoren, die zur Stabilisierung bzw. zur Lokalisation verwendet werden.  


\begin{table}[H]
	\caption["Ubersicht Sensoren]{Sensoren, die im Ballbot verbaut wurden} \label{tab:sensoren}
  		\begin{center}				
		\begin{longtable}{|c||c|c|c|}
		\hline 
  		\multicolumn{4}{|c|}{�bersicht Sensoren} \\ 
  		\hline
  		Kategorie & Stabilisierung & \multicolumn{2}{c|}{Lokalisierung/Interaktion} \\ 
  		\hline
  		\hline
  		Bezeichnung & IMU & Kamera & Laser Distance Sensor\\
  		& MPU-9250 & Intel Realsense R200 & LDS-01\\
  		\hline	\hline
  		
  		 & & & \\
         Taktfrequenz: & 4Hz - 8kHz & 2.5GHz & 1,8 kHz\\
         & & & \\
         Features: & Gyroskop & IR Laser Projector System & DA: 120mm - 3500mm\\
                  & Accelerometer & Full HD RGB Color Stream & Scan Rate: 300$\pm$10 rpm\\
                  & Magnetometer & Onboard Imaging ASIC & Angular Resolution: $1^{\circ}$\\	
  		\hline
		\end{longtable}
  		\end{center}
\end{table} 


\subsection{Motoren} \label{sec:Motoren}

\begin{figure}[H]%
	\centering
	{\includegraphics[scale=6.4]{Bilder/Michael/Motoren.jpg} }
	\caption{Motorenspezifikationen}
	\label{fig:motorenspezifikationen}
\end{figure}


Im TurtleBot3-Paket enthalten sind weiterhin auch die Motoren, welche f"ur den Ballbotaufbau verwendet werden. Es handelt sich dabei um Dynamixel Servomotoren der Baureihe XM430-W350-T. Die Motoren haben sich als sehr robust erwiesen und k"onnen durch eine hohe "Ubersetzung von $i=1:353.5$ ein variables ausgangsseitiges Drehmoment von bis zu $3.8$\,Nm bei $11.1$\,V \cite{XM430} erzeugen. Dies ist f"ur die gegebenen Systemkonfiguration mehr als ausreichend. Weiterhin werden von Dynamixel bereits verschiedene Betriebsarten bereitgestellt, in denen die Motoren betrieben werden k"onnen. So bieten die Motoren neben einer Positionsregelung, einer Geschwindigkeitsregelung auch eine Stromregelung. Durch die lineare Abh"angikeit zwischen Drehmoment und Strom kann somit das abtriebseitige Drehmoment geregelt werden.
 
Vom OpenCR-Board werden mittels dem RS485-Protokoll sogennante Units �bertragen. Die Units werden dann intern im Motor in einen realen Sollstrom umgewandelt. Der Strom kann �ber die Formel 

\begin{align} \label{eq:CurrentUnit}
I &= k_{unit}\cdot Units, && \text{mit}  & k_{unit} &= 2,69\cdot 10^{-3} \frac{A}{Unit}
\end{align}

berechnet werden. Anschlie�end kann mit Hilfe der Drehmoment(Nm)-Strom(A)-Kennlinie aus dem Datenblatt \cite{XM430} die spezifische Motorkonstante zu $k_{motor} = 1,63$\,Nm/A bestimmt werden.  
  bestimmt.  Dadurch kann das vom Motor erzeugte Drehmoment aus den Units mittels der Formel 
  \begin{align} \label{eq:CurrentTorque}
  M &= k_{motor} \cdot I \\
    &= k_{motor}\cdot k_{unit} \cdot Units
  \end{align}
  berechnet werden. Insgesamt ergibt sich folgender Gesamtumrechnungsfaktor 
    \begin{align} \label{eq:CurrentTorque}
    k &= k_{motor}\cdot k_{unit} = 228\cdot \frac{Nm}{Unit} 
    \end{align}
    
In der technischen Umsetzung hat sich herausgestellt, dass die Motoren zueinander ein unterschiedliches Verhalten aufweisen und somit die Konstante $k$ nicht f�r alle drei Motoren verwendet werden kann. Daher wurde die Konstante $k$ experimentell bestimmt.
Hierzu wird ein Pr�fhebel der L�nge $l_{pruef}$ f�r die Motoren konstruiert, der w�hrend des Betriebs der Motoren auf eine Waage dr�ckt. Die Motoren werden mit einer bestimmten Folge von dimensionslosen Einheiten angesteuert und jeweils das angezeigte Gewicht der Waage $m$ notiert.
Mit der Beziehung
\begin{align} \label{eq:FMG}
F &= m \cdot g \nonumber
\end{align}
eingesetzt in
\begin{align} 
 M &= F \cdot l_{\text{pruef}}
\end{align}
kann das resultierende Drehmoment $M$ berechnet werden und somit die Konstante $k$ f�r jeden einzelnen Motor.

Bei den Messungen wurde festgestellt, dass das von den Motoren erzeugte Drehmoment einen Drift\footnote{Unter einem Drift ist hier gemeint, dass bei konstantem Anlegen von Units (Stromwert) das Drehmoment �ber die Zeit abnimmt.} aufweist. Dieser Drift konnte reduziert werden, in dem man statt einmaligen Drehmomentbefehlen, die Befehle in Form einer Pulsung wiederkehrend an den Motor �bergeben hat. Die neuen $k$-Faktoren sind �ber die Ergebnisse der Messungen, dargestellt in den Abbildungen \ref{fig:Motorkonstante1}, \ref{fig:Motorkonstante2} und \ref{fig:Motorkonstante3}, mittels der Methode der kleinsten Quadrate berechnet worden. Aus diesen Abbildungen ist zu erkennen, dass sich die Drehmoment-Unit-Konstanten f�r die einzelnen baugleichen Motoren unterscheiden. Dieses Verhalten wurde bei der Implementierung in Kapitel \ref{ch:Implementierung} ber�cksichtigt.
\begin{figure}[!htbp]%
	\centering
	{\includegraphics[scale=0.55]{Bilder/Michael/Motor1.png} }
	\caption{Bestimmung der Drehmoment-Unit-Konstante f�r Motor 1}
	\label{fig:Motorkonstante1}
\end{figure}

\begin{figure}[!htbp]%
	\centering
	{\includegraphics[scale=0.55]{Bilder/Michael/Motor2.png} }
	\caption{Bestimmung der Drehmoment-Unit-Konstante f�r Motor 2}
	\label{fig:Motorkonstante2}
\end{figure}

\begin{figure}[!htbp]%
	\centering
	{\includegraphics[scale=0.55]{Bilder/Michael/Motor3.png} }
	\caption{Bestimmung der Drehmoment-Unit-Konstante f�r Motor 3}
	\label{fig:Motorkonstante3}
\end{figure}      
            
\section{Ballbot-Design} \label{sec:entwurf}
Die mechanische Systemkonfiguration kann in drei Teile untergliedert werden. Dies ist zum einen der Oberbau (Kap.\ref{sec:oberbau}), der Unterbau (Kap.\ref{sec:unterbau}) sowie der Ball (Kap.\ref{sec:ball}), auf dem das Gesamtsystem bestehend aus Ober- und Unterbau balancieren wird. Nachfolgende Kapitel dokumentieren die Entwicklungsprozesse der einzelnen Teile. 

\subsection{Oberbau} \label{sec:oberbau}

\begin{figure}[H]%
	\centering
	{\includegraphics[scale=6.4]{Bilder/Michael/Oberbau.jpg} }
	\caption{Oberbau, bestehend aus vier Etagen}
	\label{fig:oberbau}
\end{figure}

Die Konfiguration des Oberbaus hat sich experimentell unter Ber�cksichtigung der in Kapitel \ref{ch:Modellierung} hergeleiteten Beziehungen ergeben. In der untersten Ebene ist die Batterie platziert. Dar�ber folgt eine Ebene mit einem Up-Level Computer, der f�r rechenintensiver Aufgaben wie der Lokalisation ben�tigt wird. Dieser wird jedoch in dieser Arbeit nicht weiter ber�cksichtigt. Anschlie�end folgt das OpenCR-Board mit den dar�ber platzierten Sensoren zur Lokalisierung des Roboters im Raum, vgl. Abbildung \ref{fig:oberbau}. 


\subsection{Unterbau} \label{sec:unterbau}

\begin{figure}[htb!]%
	\centering
	{\includegraphics[scale=6.4]{Bilder/Michael/Unterbau.jpg} }
	\caption{Unterbau}
	\label{fig:Unterbau}
\end{figure}

Die Aufgabe des Unterbaus besteht in der Positionierung der Antriebe. Dabei sollte dieser m�glichst leicht und zugleich stabil konstruiert werden. 

Damit das Drehmoment der Motoren optimal auf den Ball �bertragen werden kann, m�ssen die R�der gegen�ber der z-Achse\footnote{Die z-Achse ist die zum Ballbot vertikale Achse.} des Ballbots einen Winkel von $\alpha=45^{\circ}$ einschlie�en. Weiterhin wurden die R�der um einen Winkel von $\beta=120^{\circ}$ zueinander versetzt befestigt. Diese konstruktionsbedingten Winkel sind in Abbildung \ref{fig:konstruktions�bersicht} dargestellt.

\begin{figure}[H]%
	\centering
	{ \scalebox{1}{{\chapter{Konstruktion} \label{ch:konstruktion}

\section{Konzeptionierung} \label{sec:konzeptionierung}

\begin{figure}[H]%
	\centering
	{\includegraphics[scale=6.4]{Bilder/Michael/TUD_Ballbot_V30_MS_mit_Raeder_BAll.jpg} }
	\caption{Oberbau, bestehend aus vier Etagen}
	\label{fig:oberbau}
\end{figure}

F�r den Entwurf eines auf einem Ball balancierenden Roboters kann das System zur Vereinfachung als ein inverses Pendel betrachtet werden, dessen Drehachse auf einer beweglichen Plattform gelagert ist. Dies ist in Abbildung \ref{fig:invPendulum} beispielhaft dargestellt. Unter dieser Annahme k�nnen erste Absch�tzungen �ber das dynamische Verhalten getroffen werden. Aus der Anschauung heraus kann direkt gefolgert werden, dass der Pendelarm mit der L�nge $l$ einen Einfluss auf die Pendeldynamik des Systems haben muss. Leitet man nun wie in \cite{pendel}, unter Verwendung der Momentenbilanz das nichtlineare Differenzialgleichungssstem zweiter Ordnung des Auslenkwinkels $\vartheta$ sowie der Schlittenstrecke $x_w$ her, so erh�lt man die Gleichungen: 

        \begin{align} \label{eq:InvPendelThetaDotDot}
            \ddot{\vartheta} &= \frac{l\cdot m}{I}\cdot [-\cos{\vartheta} \cdot \ddot{x}_{w}+g\cdot  \sin{\vartheta}] ,\\   
            \label{eq:InvPendelXDotDot}
            \ddot{x}_w &= \frac{F_{motor}-F_{Reib}}{M+m} + \frac{m\cdot l}{M+m}\dot{\vartheta}^2 \cdot \sin{\vartheta} - \ddot{\vartheta} . 
        \end{align}


Setzt man nun das Massentr�gheitsmoment des inversen Pendels

        \begin{align} \label{eq:tr�gheitsmoment}
            I &= \frac{4}{3}m\cdot l^2 ,   
        \end{align}
in Gl.\,\ref{eq:InvPendelThetaDotDot} ein, so ergibt sich:  

        \begin{align} \label{eq:InvPendelThetaDotDot_Neu}
            \ddot{\vartheta} &= \frac{3}{4\cdot l}\cdot[-\cos{\vartheta} \cdot \ddot{x}_w+g \cdot  \sin{\vartheta}] . 
        \end{align}

\begin{figure}[H]%
	\centering
	{ \scalebox{1}{{\input{./Bilder/Michael/Schlitten.pdf_tex}}}}%
	\caption{Darstellung eines Ballbots als inverses Pendel}%
	\label{fig:invPendulum}%
\end{figure}%\newline

Anhand der Gleichung (\ref{eq:InvPendelThetaDotDot_Neu}) l�sst sich wie bereits vermutet eine Abh"angikeit der Winkelbeschleunigung $\ddot{\vartheta}$ von der Hebelarml"ange $l$ erkennen. Da sich die Hebelarml�nge jedoch aus der Distanz zwischen Drehachse und Massenschwerpunkt zusammensetzt, bestimmt die Verteilung der Ballbotmasse die Hebelarml�nge des inversen Pendels. Durch das Vergr��ern der Hebelarml�nge $l$ kann so die Winkelbeschleunigung $\ddot{\vartheta}$ f�r kleine Winkelauslenkungen $\Delta \vartheta$ verringert werden. Dadurch ist das System leichter zu stabilisieren. Bei Betrachtung der Gleichung (\ref{eq:InvPendelXDotDot}) l�sst sich jedoch erkennen, dass mit einer Verl�ngerung des Hebelarms $l$ eine gr��ere Beschleunigung $\ddot{x}_w$  des Schlittens einhergehen muss, die durch entsprechend starke Drehmomente realisiert werden kann.

Wendet man dieses Prinzip eines gro�en Hebelarms auf den Ballbot an, so stellt sich jedoch heraus, dass es nur bedingt anwendbar ist. Ursache hierf�r ist die Schnittstelle zwischen omnidirektionalen R�dern und Ball, da hier aufgrund des Reibkoeffizienten nur ein begrenztes Drehmoment �bertragen werden kann. Der Reibkoeffizient begrenzt somit die gew�nschte Sollbeschleunigung des Balles. 

Da eine optimale Auslegung der Pendelarml�nge rechnerisch nur schwer zu bestimmen ist, w�re es w�nschenswert, die Hebelarml�nge durch eine flexible Hardware ver�ndern zu k�nnen. Durch das Hinzuf�gen bzw. Entfernen von weiteren Ebenen im Aufbau des Ballbots kann der Massenschwerpunkt und somit die Pendelarml�nge des Ballbots variabel eingestellt werden. So kann die optimale Konfiguration sehr einfach experimentell bestimmt werden. 

Als Hardware f�r den Ballbot hat man sich daher f�r das TurtleBot3-Paket in der Burger-Variante entschieden. Auf diese soll im Kapitel \ref{sec:Hardware} tiefer eingegangen werden. Bei der Wahl des Balles sowie der Antriebsanordnung hat man sich die Konzepte bereits funktionierender Ballbots als Grundlage hergenommen. Die folgenden Kapitel gehen dabei n�her auf die einzelnen Konstruktionsschritte ein. 

\section{Verwendete Hardware} \label{sec:Hardware}
 Beim Entwurf des Ballbots wurde, wie bereits erw�hnt, auf die leistungsf"ahige Hardware eines TurtleBot3 in der \glqq Burger\grqq-Variante zur"uck gegriffen. Dieses Paket umfasst neben einem robusten mechanischen Aufbau auch alle grundlegenden elektronischen Komponenten wie Sensoren, Mikrocontrollern und Antriebseinheiten, die f�r die Entwicklung eines auf einem Ball balancierenden Roboters ben�tigt werden. Die wichtigsten Komponenten werden in den nachfolgenden Unterkapitel n�her erl�utert.

\subsection{OpenCR-Board} \label{sec:openCr}

Das im Projekt verwendete OpenCR-Board (in der Version 1.0), wie in Abbildung \ref{fig:opencr_board} dargestellt, ist ein Mikrocontroller, der f�r Roboterprojekte in der Gr��e des TurtleBot3-Pakets perfekt geeignet ist. Das gesamte Board, von der Hardware bis zur Software, ist durch Open-Source-Lizenzen frei verf�gbar. Es bietet mit insgesamt 10 verschiedenen Kommunikationsstellen und einer integrierten inertialen Messeinheit (IMU)\footnote{Die verwendete IMU ist die MPU-9250 siehe \url{https://www.invensense.com/products/motion-tracking/9-axis/mpu-9250/}.} bei der Entwicklung von Robotersystemen. Dar�ber hinaus ist das OpenCR-Board mit einem leistungsstarken ARM Cortex-M7 Prozessor ausgestattet.

\begin{figure}[H]%H !htbp
	\centering
	{\includegraphics[scale=0.45]{Bilder/Michael/opencr.png} }
	\caption{�bersicht der Schnittstellen des OpenCR-Boards \cite{openCR}}
	\label{fig:opencr_board}
\end{figure}

%https://github.com/ROBOTIS-GIT/emanual/blob/master/docs/en/platform/turtlebot3/appendix_opencr1_0.md

\subsection{Sensorik} \label{sec:sensoren}
Am Ballbot wurden mehrere Sensoren verbaut, die f�r verschiedene Aufgabenbereiche ben�tigt werden. Tabelle \ref{tab:sensoren} zeigt dabei eine �bersicht der Sensoren, die zur Stabilisierung bzw. zur Lokalisation verwendet werden.  


\begin{table}[H]
	\caption["Ubersicht Sensoren]{Sensoren, die im Ballbot verbaut wurden} \label{tab:sensoren}
  		\begin{center}				
		\begin{longtable}{|c||c|c|c|}
		\hline 
  		\multicolumn{4}{|c|}{�bersicht Sensoren} \\ 
  		\hline
  		Kategorie & Stabilisierung & \multicolumn{2}{c|}{Lokalisierung/Interaktion} \\ 
  		\hline
  		\hline
  		Bezeichnung & IMU & Kamera & Laser Distance Sensor\\
  		& MPU-9250 & Intel Realsense R200 & LDS-01\\
  		\hline	\hline
  		
  		 & & & \\
         Taktfrequenz: & 4Hz - 8kHz & 2.5GHz & 1,8 kHz\\
         & & & \\
         Features: & Gyroskop & IR Laser Projector System & DA: 120mm - 3500mm\\
                  & Accelerometer & Full HD RGB Color Stream & Scan Rate: 300$\pm$10 rpm\\
                  & Magnetometer & Onboard Imaging ASIC & Angular Resolution: $1^{\circ}$\\	
  		\hline
		\end{longtable}
  		\end{center}
\end{table} 


\subsection{Motoren} \label{sec:Motoren}

\begin{figure}[H]%
	\centering
	{\includegraphics[scale=6.4]{Bilder/Michael/Motoren.jpg} }
	\caption{Motorenspezifikationen}
	\label{fig:motorenspezifikationen}
\end{figure}


Im TurtleBot3-Paket enthalten sind weiterhin auch die Motoren, welche f"ur den Ballbotaufbau verwendet werden. Es handelt sich dabei um Dynamixel Servomotoren der Baureihe XM430-W350-T. Die Motoren haben sich als sehr robust erwiesen und k"onnen durch eine hohe "Ubersetzung von $i=1:353.5$ ein variables ausgangsseitiges Drehmoment von bis zu $3.8$\,Nm bei $11.1$\,V \cite{XM430} erzeugen. Dies ist f"ur die gegebenen Systemkonfiguration mehr als ausreichend. Weiterhin werden von Dynamixel bereits verschiedene Betriebsarten bereitgestellt, in denen die Motoren betrieben werden k"onnen. So bieten die Motoren neben einer Positionsregelung, einer Geschwindigkeitsregelung auch eine Stromregelung. Durch die lineare Abh"angikeit zwischen Drehmoment und Strom kann somit das abtriebseitige Drehmoment geregelt werden.
 
Vom OpenCR-Board werden mittels dem RS485-Protokoll sogennante Units �bertragen. Die Units werden dann intern im Motor in einen realen Sollstrom umgewandelt. Der Strom kann �ber die Formel 

\begin{align} \label{eq:CurrentUnit}
I &= k_{unit}\cdot Units, && \text{mit}  & k_{unit} &= 2,69\cdot 10^{-3} \frac{A}{Unit}
\end{align}

berechnet werden. Anschlie�end kann mit Hilfe der Drehmoment(Nm)-Strom(A)-Kennlinie aus dem Datenblatt \cite{XM430} die spezifische Motorkonstante zu $k_{motor} = 1,63$\,Nm/A bestimmt werden.  
  bestimmt.  Dadurch kann das vom Motor erzeugte Drehmoment aus den Units mittels der Formel 
  \begin{align} \label{eq:CurrentTorque}
  M &= k_{motor} \cdot I \\
    &= k_{motor}\cdot k_{unit} \cdot Units
  \end{align}
  berechnet werden. Insgesamt ergibt sich folgender Gesamtumrechnungsfaktor 
    \begin{align} \label{eq:CurrentTorque}
    k &= k_{motor}\cdot k_{unit} = 228\cdot \frac{Nm}{Unit} 
    \end{align}
    
In der technischen Umsetzung hat sich herausgestellt, dass die Motoren zueinander ein unterschiedliches Verhalten aufweisen und somit die Konstante $k$ nicht f�r alle drei Motoren verwendet werden kann. Daher wurde die Konstante $k$ experimentell bestimmt.
Hierzu wird ein Pr�fhebel der L�nge $l_{pruef}$ f�r die Motoren konstruiert, der w�hrend des Betriebs der Motoren auf eine Waage dr�ckt. Die Motoren werden mit einer bestimmten Folge von dimensionslosen Einheiten angesteuert und jeweils das angezeigte Gewicht der Waage $m$ notiert.
Mit der Beziehung
\begin{align} \label{eq:FMG}
F &= m \cdot g \nonumber
\end{align}
eingesetzt in
\begin{align} 
 M &= F \cdot l_{\text{pruef}}
\end{align}
kann das resultierende Drehmoment $M$ berechnet werden und somit die Konstante $k$ f�r jeden einzelnen Motor.

Bei den Messungen wurde festgestellt, dass das von den Motoren erzeugte Drehmoment einen Drift\footnote{Unter einem Drift ist hier gemeint, dass bei konstantem Anlegen von Units (Stromwert) das Drehmoment �ber die Zeit abnimmt.} aufweist. Dieser Drift konnte reduziert werden, in dem man statt einmaligen Drehmomentbefehlen, die Befehle in Form einer Pulsung wiederkehrend an den Motor �bergeben hat. Die neuen $k$-Faktoren sind �ber die Ergebnisse der Messungen, dargestellt in den Abbildungen \ref{fig:Motorkonstante1}, \ref{fig:Motorkonstante2} und \ref{fig:Motorkonstante3}, mittels der Methode der kleinsten Quadrate berechnet worden. Aus diesen Abbildungen ist zu erkennen, dass sich die Drehmoment-Unit-Konstanten f�r die einzelnen baugleichen Motoren unterscheiden. Dieses Verhalten wurde bei der Implementierung in Kapitel \ref{ch:Implementierung} ber�cksichtigt.
\begin{figure}[!htbp]%
	\centering
	{\includegraphics[scale=0.55]{Bilder/Michael/Motor1.png} }
	\caption{Bestimmung der Drehmoment-Unit-Konstante f�r Motor 1}
	\label{fig:Motorkonstante1}
\end{figure}

\begin{figure}[!htbp]%
	\centering
	{\includegraphics[scale=0.55]{Bilder/Michael/Motor2.png} }
	\caption{Bestimmung der Drehmoment-Unit-Konstante f�r Motor 2}
	\label{fig:Motorkonstante2}
\end{figure}

\begin{figure}[!htbp]%
	\centering
	{\includegraphics[scale=0.55]{Bilder/Michael/Motor3.png} }
	\caption{Bestimmung der Drehmoment-Unit-Konstante f�r Motor 3}
	\label{fig:Motorkonstante3}
\end{figure}      
            
\section{Ballbot-Design} \label{sec:entwurf}
Die mechanische Systemkonfiguration kann in drei Teile untergliedert werden. Dies ist zum einen der Oberbau (Kap.\ref{sec:oberbau}), der Unterbau (Kap.\ref{sec:unterbau}) sowie der Ball (Kap.\ref{sec:ball}), auf dem das Gesamtsystem bestehend aus Ober- und Unterbau balancieren wird. Nachfolgende Kapitel dokumentieren die Entwicklungsprozesse der einzelnen Teile. 

\subsection{Oberbau} \label{sec:oberbau}

\begin{figure}[H]%
	\centering
	{\includegraphics[scale=6.4]{Bilder/Michael/Oberbau.jpg} }
	\caption{Oberbau, bestehend aus vier Etagen}
	\label{fig:oberbau}
\end{figure}

Die Konfiguration des Oberbaus hat sich experimentell unter Ber�cksichtigung der in Kapitel \ref{ch:Modellierung} hergeleiteten Beziehungen ergeben. In der untersten Ebene ist die Batterie platziert. Dar�ber folgt eine Ebene mit einem Up-Level Computer, der f�r rechenintensiver Aufgaben wie der Lokalisation ben�tigt wird. Dieser wird jedoch in dieser Arbeit nicht weiter ber�cksichtigt. Anschlie�end folgt das OpenCR-Board mit den dar�ber platzierten Sensoren zur Lokalisierung des Roboters im Raum, vgl. Abbildung \ref{fig:oberbau}. 


\subsection{Unterbau} \label{sec:unterbau}

\begin{figure}[htb!]%
	\centering
	{\includegraphics[scale=6.4]{Bilder/Michael/Unterbau.jpg} }
	\caption{Unterbau}
	\label{fig:Unterbau}
\end{figure}

Die Aufgabe des Unterbaus besteht in der Positionierung der Antriebe. Dabei sollte dieser m�glichst leicht und zugleich stabil konstruiert werden. 

Damit das Drehmoment der Motoren optimal auf den Ball �bertragen werden kann, m�ssen die R�der gegen�ber der z-Achse\footnote{Die z-Achse ist die zum Ballbot vertikale Achse.} des Ballbots einen Winkel von $\alpha=45^{\circ}$ einschlie�en. Weiterhin wurden die R�der um einen Winkel von $\beta=120^{\circ}$ zueinander versetzt befestigt. Diese konstruktionsbedingten Winkel sind in Abbildung \ref{fig:konstruktions�bersicht} dargestellt.

\begin{figure}[H]%
	\centering
	{ \scalebox{1}{{\input{./Bilder/Michael/Konstruktion.pdf_tex}}}}%
	\caption{Links: Seitenansicht der Ball-Rad- Konfiguration. \\Rechts: Draufsicht der Ball-Rad -Konfiguration  }%
	\label{fig:konstruktions�bersicht}
\end{figure}%\newline

Unter Ber�cksichtigung dieser Vorgaben ist der Unterbau entwickelt worden. Die daf�r notwendigen Konstruktionen sind mit dem CAD-Tool SolidEdge durchgef�hrt und anschlie�end auf einen 3D-Drucker �bertragen und gedruckt worden. Auf diesem Weg war es m�glich schnell und kosteng�nstig neue Ideen und Designs umzusetzen. Diese Schritte werden im Kapitel \ref{sec:SolidEdge} erl�utert. Zun�chst soll jedoch noch auf die Wahl des Balls eingegangen werden. 

\subsection{Wahl des Balls} \label{sec:ball}
Der Ball stellt eine weitere Schl�sselkomponente bei der Entwicklung eines Ballbots dar. Im Verlauf der Balancierungstest sind einige verschiede B�lle zum Einsatz gekommen. Die Suche nach einem optimalen Ball hat sich dabei als �u�erst schwierig herausgestellt, da mehrere Parameter des Balls Einfluss auf das Balancierverhalten haben. Neben der Gr��e und Kompressibilit�t spielen besonders auch der Reibkoeffizient und das Tr�gheitsmoment eine gro�e Rolle. 

Die Gr��e des Balles muss auf die Auslegung des Unterbaus ausgerichtet werden, da sich sonst ein falscher Winkel $\alpha$ f�r das reale System ergibt. Weiterhin ist bei vielen B�llen immer wieder Schlupf zwischen Omniwheels und Ball aufgetreten. Sobald Schlupf auftrat, konnte der Ballbot durch die Regelung nicht mehr stabilisiert werden. Dies gilt es daher unbedingt zu vermeiden. Zudem d�rfen keine Verformungen des Balles bei Belastungen durch den Ballbot auftreten, da sich der Ballbot sonst aufschaukelt und das Regelverhalten einer unged�mpften Schwingung gleicht.
Am Ende der Experimente hat sich ein Handball der Gr��e II auf einer d�nnen Schaumstoffmatte als bester Kompromiss herausgestellt. Die Schaumstoffmatte wurde dabei zur D�mpfung der Ballbewegung verwendet.

\section{Konstruktion mittels SolidEdge} \label{sec:SolidEdge}
SolidEdge ist eine Computer gest�tztes Design(CAD)-Software, die ein rechnergest�tztes Konstruieren einzelner Bauteile sowie ganzer Baugruppen eines Produktes, einer Maschine, etc. erm�glicht. Weiterhin gibt es auch Tools zur Bauteil-Optimierung und Simulation von Str�mungen. SolidEdge wird f�r Studenten kostenlos von der Siemens Industry Software GmbH zur Verf�gung gestellt und kann unter Angabe der pers�nlichen Daten heruntergeladen werden\footnote{\url{https://www.plm.automation.siemens.com/plmapp/education/solid-edge/en_us/free-software/student} (Stand: 07.02.2018)}. \\

\subsection{SolidEdge - Eine kurze Einf�hrung} \label{whatIs}
Bei der Konstruktion der einzelnen Unterbaukomponenten des Ballbots sind haupts�chlich zwei Funktionen von SolidEdge genutzt worden. Zum einen die Erzeugung einzelner Bauteile, worunter im Folgenden ein einzelner K�rper\footnote{Unter dem Begriff K�rper soll in diesem Anwendungsfall ein einzelnes, nicht aus mehreren Teilkomponenten bestehendes, Werkst�ck verstanden werden.} verstanden wird, sowie die Erzeugung von Baugruppen \footnote{Zusammensetzung/Montage einzelner K�rper zu einer Gruppe.}. Bei einer Baugruppe werden die Bauteile automatisch je nach Montagevorschrift zusammengef�gt und �ber sogenannte Beziehungen miteinander verbunden. Wird eine Baugruppe in eine neue Datei geladen, verh�lt sich die Baugruppe durch die definierten Beziehungen wie ein einzelner K�rper. Auf diesem Weg k�nnen in eine Baugruppe auch andere Baugruppen geladen und miteinander in Beziehung gestellt werden. Das macht ein schrittweises und �bersichtliches Zusammensetzen der Gesamtkonstruktion m�glich.  

\subsection{SolidEdge - Umsetzung} \label{subsec:Umsetzung}
Es soll nun ein Einblick in die Vorgehensweise bei der Konstruktion des Unterbaus mit SolidEdge gegeben werden. Abbildung \ref{fig:Unterbau} zeigt die Baugruppe \glqq Unterbau\grqq. Folgende Kriterien galt es dabei zu erf�llen: Die Baugruppe sollte aus mehreren K�rpern besteht, sodass ein einfaches Montieren der einzelnen Komponenten m�glich ist. Weiterhin sollte eine flexible Konstruktion entwickelt werden, um den Ballbot auf B�llen mit verschiedenen Radien testen zu k�nnen. Dies wurde durch eine Teleskopsystem erm�glicht, sodass die Motoren inklusive R�der radial nach au�en bzw. innen geschoben werden k�nnen. Dies ist in Abbildung \ref{fig:Motorenhalter} dargestellt. 
\begin{figure}[!htbp]%
	\centering
	{\includegraphics[scale=4.9]{Bilder/Michael/Traeger_Halterung_Explosion.jpg} }
	\caption{Teleskopsystem bestehend aus Tr�gerplatte(links) und Motorenhalter(rechts)}
	\label{fig:Motorenhalter}
\end{figure}

R�cksicht wurde weiterhin auf eine stabile Integration der Motoren in die Gesamtkonstruktion genommen. So soll bei den wirkenden Kr�ften und Momenten, Bewegungen der Unterbaukomponenten relativ zu einander vermieden werden. Integriert wurden die Motoren daher �ber vier M3 Zylinderkopfschrauben an den sogenannten Motorenhaltern, wie in Abbildung \ref{fig:antrieb} dargestellt. Die F�hrungsschiene des Motorenhalters  wird in eine F�hrungsnut der Tr�gerplatte passgenau eingef�hrt und verschraubt.


\begin{figure}[!htbp]%
	\centering
	{\includegraphics[scale=4.8]{Bilder/Michael/Motor_Mitnehmer_Halterung_Explosion.jpg} }
	\caption{Antriebsvorrichtung bestehend aus Mitnehmer (unten), Motor (mitte) und Motorenhalter (oben). }
	\label{fig:antrieb}
\end{figure}

Um die R�der, dargestellt in Abbildung \ref{fig:wheel} an die Motorwellen zu befestigen, sind spezielle Mitnehmer konstruiert worden, die in Abbildung \ref{fig:antrieb} dargestellt sind.

Wichtige Kriterien, die es hierbei zu erf�llen gab, waren eine spielfrei �bertragung der Momente und Kr�fte sowie eine m�glichst kurze Distanz zwischen den R�dern und den Motoren.

Nach einigen Tests hat sich herausgestellt, dass die gew�nschte Stabilit�t es Unterbaus nicht gegeben war. Die Konstruktion musste also noch verst�rkt werden. Um die n�tige Stabilit�t herzustellen, sind die Motorenhalter um eine Anflanschfl�che erweitert worden. So k�nnen die Motorenhalter �ber ein Y-f�rmiges Verbindungsst�ck, dargestellt in Abbildung \ref{fig:kreisring}, miteinander gekoppelt werden. Hierbei wurde eine flexible Anpassung an unterschiedlichen Ballradien ber�cksichtigt.
\begin{figure}[!htbp]%
	\centering
	{\includegraphics[scale=6.4]{Bilder/Michael/Wheel2_Double.jpg} }
	\caption{Modell eines omnidirektionalen Rades}
	\label{fig:wheel}
\end{figure}

\begin{figure}[!htbp]%
	\centering
	{\includegraphics[scale=4.8]{Bilder/Michael/Kreisring_Stabi.jpg} }
	\caption{Y-f�rmiges Verbindungsst�ck des Unterbaus.}
	\label{fig:kreisring}
\end{figure}

\section{Fertigung der Komponenten} \label{sec:Fertigung}
Neben dem Zugang zu einem professionellen CAD-Tool bestand auch Zugang zu einem 3D-Drucker der Firma Oktoprint. Dies erm�glichte eine kosteng�nstige und schnelle Fertigung der konstruierten Bauteile. Verwendete wurde dabei das Filament Polylactide (PLA). PLA ist ein sehr verbreiteter biokompatibler Kunstoff, der eine hohe Oberfl�chenh�rte, hohe Steifigkeit und eine hohe Zugfestigkeit bietet \cite{pla}. 

Der fertige Entwurf wurde schlie�lich als .stl exportiert und in das 3D-Druckprogramm CURA geladen. Mit CURA konnte anschlie�end die F�lldichte, die Qualit�t, Druckgeschwindigkeit und der Einsatz von St�tzhilfen f�r den jeweiligen Druck eingestellt werden. Die gew�nschten Genauigkeiten konnten mittels dieser Fertigungsweise eingehalten werden. So hat der Druck auch an kritischen Stellen wie der Einschubverbindung zwischen Tr�gerplatte und Motorenhalter �berzeugt. Es war nach Einschub und Verschraubung der Motorenhalter mit der Tr�gerplatte kein Spiel vorhanden. }}}%
	\caption{Links: Seitenansicht der Ball-Rad- Konfiguration. \\Rechts: Draufsicht der Ball-Rad -Konfiguration  }%
	\label{fig:konstruktions�bersicht}
\end{figure}%\newline

Unter Ber�cksichtigung dieser Vorgaben ist der Unterbau entwickelt worden. Die daf�r notwendigen Konstruktionen sind mit dem CAD-Tool SolidEdge durchgef�hrt und anschlie�end auf einen 3D-Drucker �bertragen und gedruckt worden. Auf diesem Weg war es m�glich schnell und kosteng�nstig neue Ideen und Designs umzusetzen. Diese Schritte werden im Kapitel \ref{sec:SolidEdge} erl�utert. Zun�chst soll jedoch noch auf die Wahl des Balls eingegangen werden. 

\subsection{Wahl des Balls} \label{sec:ball}
Der Ball stellt eine weitere Schl�sselkomponente bei der Entwicklung eines Ballbots dar. Im Verlauf der Balancierungstest sind einige verschiede B�lle zum Einsatz gekommen. Die Suche nach einem optimalen Ball hat sich dabei als �u�erst schwierig herausgestellt, da mehrere Parameter des Balls Einfluss auf das Balancierverhalten haben. Neben der Gr��e und Kompressibilit�t spielen besonders auch der Reibkoeffizient und das Tr�gheitsmoment eine gro�e Rolle. 

Die Gr��e des Balles muss auf die Auslegung des Unterbaus ausgerichtet werden, da sich sonst ein falscher Winkel $\alpha$ f�r das reale System ergibt. Weiterhin ist bei vielen B�llen immer wieder Schlupf zwischen Omniwheels und Ball aufgetreten. Sobald Schlupf auftrat, konnte der Ballbot durch die Regelung nicht mehr stabilisiert werden. Dies gilt es daher unbedingt zu vermeiden. Zudem d�rfen keine Verformungen des Balles bei Belastungen durch den Ballbot auftreten, da sich der Ballbot sonst aufschaukelt und das Regelverhalten einer unged�mpften Schwingung gleicht.
Am Ende der Experimente hat sich ein Handball der Gr��e II auf einer d�nnen Schaumstoffmatte als bester Kompromiss herausgestellt. Die Schaumstoffmatte wurde dabei zur D�mpfung der Ballbewegung verwendet.

\section{Konstruktion mittels SolidEdge} \label{sec:SolidEdge}
SolidEdge ist eine Computer gest�tztes Design(CAD)-Software, die ein rechnergest�tztes Konstruieren einzelner Bauteile sowie ganzer Baugruppen eines Produktes, einer Maschine, etc. erm�glicht. Weiterhin gibt es auch Tools zur Bauteil-Optimierung und Simulation von Str�mungen. SolidEdge wird f�r Studenten kostenlos von der Siemens Industry Software GmbH zur Verf�gung gestellt und kann unter Angabe der pers�nlichen Daten heruntergeladen werden\footnote{\url{https://www.plm.automation.siemens.com/plmapp/education/solid-edge/en_us/free-software/student} (Stand: 07.02.2018)}. \\

\subsection{SolidEdge - Eine kurze Einf�hrung} \label{whatIs}
Bei der Konstruktion der einzelnen Unterbaukomponenten des Ballbots sind haupts�chlich zwei Funktionen von SolidEdge genutzt worden. Zum einen die Erzeugung einzelner Bauteile, worunter im Folgenden ein einzelner K�rper\footnote{Unter dem Begriff K�rper soll in diesem Anwendungsfall ein einzelnes, nicht aus mehreren Teilkomponenten bestehendes, Werkst�ck verstanden werden.} verstanden wird, sowie die Erzeugung von Baugruppen \footnote{Zusammensetzung/Montage einzelner K�rper zu einer Gruppe.}. Bei einer Baugruppe werden die Bauteile automatisch je nach Montagevorschrift zusammengef�gt und �ber sogenannte Beziehungen miteinander verbunden. Wird eine Baugruppe in eine neue Datei geladen, verh�lt sich die Baugruppe durch die definierten Beziehungen wie ein einzelner K�rper. Auf diesem Weg k�nnen in eine Baugruppe auch andere Baugruppen geladen und miteinander in Beziehung gestellt werden. Das macht ein schrittweises und �bersichtliches Zusammensetzen der Gesamtkonstruktion m�glich.  

\subsection{SolidEdge - Umsetzung} \label{subsec:Umsetzung}
Es soll nun ein Einblick in die Vorgehensweise bei der Konstruktion des Unterbaus mit SolidEdge gegeben werden. Abbildung \ref{fig:Unterbau} zeigt die Baugruppe \glqq Unterbau\grqq. Folgende Kriterien galt es dabei zu erf�llen: Die Baugruppe sollte aus mehreren K�rpern besteht, sodass ein einfaches Montieren der einzelnen Komponenten m�glich ist. Weiterhin sollte eine flexible Konstruktion entwickelt werden, um den Ballbot auf B�llen mit verschiedenen Radien testen zu k�nnen. Dies wurde durch eine Teleskopsystem erm�glicht, sodass die Motoren inklusive R�der radial nach au�en bzw. innen geschoben werden k�nnen. Dies ist in Abbildung \ref{fig:Motorenhalter} dargestellt. 
\begin{figure}[!htbp]%
	\centering
	{\includegraphics[scale=4.9]{Bilder/Michael/Traeger_Halterung_Explosion.jpg} }
	\caption{Teleskopsystem bestehend aus Tr�gerplatte(links) und Motorenhalter(rechts)}
	\label{fig:Motorenhalter}
\end{figure}

R�cksicht wurde weiterhin auf eine stabile Integration der Motoren in die Gesamtkonstruktion genommen. So soll bei den wirkenden Kr�ften und Momenten, Bewegungen der Unterbaukomponenten relativ zu einander vermieden werden. Integriert wurden die Motoren daher �ber vier M3 Zylinderkopfschrauben an den sogenannten Motorenhaltern, wie in Abbildung \ref{fig:antrieb} dargestellt. Die F�hrungsschiene des Motorenhalters  wird in eine F�hrungsnut der Tr�gerplatte passgenau eingef�hrt und verschraubt.


\begin{figure}[!htbp]%
	\centering
	{\includegraphics[scale=4.8]{Bilder/Michael/Motor_Mitnehmer_Halterung_Explosion.jpg} }
	\caption{Antriebsvorrichtung bestehend aus Mitnehmer (unten), Motor (mitte) und Motorenhalter (oben). }
	\label{fig:antrieb}
\end{figure}

Um die R�der, dargestellt in Abbildung \ref{fig:wheel} an die Motorwellen zu befestigen, sind spezielle Mitnehmer konstruiert worden, die in Abbildung \ref{fig:antrieb} dargestellt sind.

Wichtige Kriterien, die es hierbei zu erf�llen gab, waren eine spielfrei �bertragung der Momente und Kr�fte sowie eine m�glichst kurze Distanz zwischen den R�dern und den Motoren.

Nach einigen Tests hat sich herausgestellt, dass die gew�nschte Stabilit�t es Unterbaus nicht gegeben war. Die Konstruktion musste also noch verst�rkt werden. Um die n�tige Stabilit�t herzustellen, sind die Motorenhalter um eine Anflanschfl�che erweitert worden. So k�nnen die Motorenhalter �ber ein Y-f�rmiges Verbindungsst�ck, dargestellt in Abbildung \ref{fig:kreisring}, miteinander gekoppelt werden. Hierbei wurde eine flexible Anpassung an unterschiedlichen Ballradien ber�cksichtigt.
\begin{figure}[!htbp]%
	\centering
	{\includegraphics[scale=6.4]{Bilder/Michael/Wheel2_Double.jpg} }
	\caption{Modell eines omnidirektionalen Rades}
	\label{fig:wheel}
\end{figure}

\begin{figure}[!htbp]%
	\centering
	{\includegraphics[scale=4.8]{Bilder/Michael/Kreisring_Stabi.jpg} }
	\caption{Y-f�rmiges Verbindungsst�ck des Unterbaus.}
	\label{fig:kreisring}
\end{figure}

\section{Fertigung der Komponenten} \label{sec:Fertigung}
Neben dem Zugang zu einem professionellen CAD-Tool bestand auch Zugang zu einem 3D-Drucker der Firma Oktoprint. Dies erm�glichte eine kosteng�nstige und schnelle Fertigung der konstruierten Bauteile. Verwendete wurde dabei das Filament Polylactide (PLA). PLA ist ein sehr verbreiteter biokompatibler Kunstoff, der eine hohe Oberfl�chenh�rte, hohe Steifigkeit und eine hohe Zugfestigkeit bietet \cite{pla}. 

Der fertige Entwurf wurde schlie�lich als .stl exportiert und in das 3D-Druckprogramm CURA geladen. Mit CURA konnte anschlie�end die F�lldichte, die Qualit�t, Druckgeschwindigkeit und der Einsatz von St�tzhilfen f�r den jeweiligen Druck eingestellt werden. Die gew�nschten Genauigkeiten konnten mittels dieser Fertigungsweise eingehalten werden. So hat der Druck auch an kritischen Stellen wie der Einschubverbindung zwischen Tr�gerplatte und Motorenhalter �berzeugt. Es war nach Einschub und Verschraubung der Motorenhalter mit der Tr�gerplatte kein Spiel vorhanden. }}}%
	\caption{Links: Seitenansicht der Ball-Rad- Konfiguration. \\Rechts: Draufsicht der Ball-Rad -Konfiguration  }%
	\label{fig:konstruktions�bersicht}
\end{figure}%\newline

Unter Ber�cksichtigung dieser Vorgaben ist der Unterbau entwickelt worden. Die daf�r notwendigen Konstruktionen sind mit dem CAD-Tool SolidEdge durchgef�hrt und anschlie�end auf einen 3D-Drucker �bertragen und gedruckt worden. Auf diesem Weg war es m�glich schnell und kosteng�nstig neue Ideen und Designs umzusetzen. Diese Schritte werden im Kapitel \ref{sec:SolidEdge} erl�utert. Zun�chst soll jedoch noch auf die Wahl des Balls eingegangen werden. 

\subsection{Wahl des Balls} \label{sec:ball}
Der Ball stellt eine weitere Schl�sselkomponente bei der Entwicklung eines Ballbots dar. Im Verlauf der Balancierungstest sind einige verschiede B�lle zum Einsatz gekommen. Die Suche nach einem optimalen Ball hat sich dabei als �u�erst schwierig herausgestellt, da mehrere Parameter des Balls Einfluss auf das Balancierverhalten haben. Neben der Gr��e und Kompressibilit�t spielen besonders auch der Reibkoeffizient und das Tr�gheitsmoment eine gro�e Rolle. 

Die Gr��e des Balles muss auf die Auslegung des Unterbaus ausgerichtet werden, da sich sonst ein falscher Winkel $\alpha$ f�r das reale System ergibt. Weiterhin ist bei vielen B�llen immer wieder Schlupf zwischen Omniwheels und Ball aufgetreten. Sobald Schlupf auftrat, konnte der Ballbot durch die Regelung nicht mehr stabilisiert werden. Dies gilt es daher unbedingt zu vermeiden. Zudem d�rfen keine Verformungen des Balles bei Belastungen durch den Ballbot auftreten, da sich der Ballbot sonst aufschaukelt und das Regelverhalten einer unged�mpften Schwingung gleicht.
Am Ende der Experimente hat sich ein Handball der Gr��e II auf einer d�nnen Schaumstoffmatte als bester Kompromiss herausgestellt. Die Schaumstoffmatte wurde dabei zur D�mpfung der Ballbewegung verwendet.

\section{Konstruktion mittels SolidEdge} \label{sec:SolidEdge}
SolidEdge ist eine Computer gest�tztes Design(CAD)-Software, die ein rechnergest�tztes Konstruieren einzelner Bauteile sowie ganzer Baugruppen eines Produktes, einer Maschine, etc. erm�glicht. Weiterhin gibt es auch Tools zur Bauteil-Optimierung und Simulation von Str�mungen. SolidEdge wird f�r Studenten kostenlos von der Siemens Industry Software GmbH zur Verf�gung gestellt und kann unter Angabe der pers�nlichen Daten heruntergeladen werden\footnote{\url{https://www.plm.automation.siemens.com/plmapp/education/solid-edge/en_us/free-software/student} (Stand: 07.02.2018)}. \\

\subsection{SolidEdge - Eine kurze Einf�hrung} \label{whatIs}
Bei der Konstruktion der einzelnen Unterbaukomponenten des Ballbots sind haupts�chlich zwei Funktionen von SolidEdge genutzt worden. Zum einen die Erzeugung einzelner Bauteile, worunter im Folgenden ein einzelner K�rper\footnote{Unter dem Begriff K�rper soll in diesem Anwendungsfall ein einzelnes, nicht aus mehreren Teilkomponenten bestehendes, Werkst�ck verstanden werden.} verstanden wird, sowie die Erzeugung von Baugruppen \footnote{Zusammensetzung/Montage einzelner K�rper zu einer Gruppe.}. Bei einer Baugruppe werden die Bauteile automatisch je nach Montagevorschrift zusammengef�gt und �ber sogenannte Beziehungen miteinander verbunden. Wird eine Baugruppe in eine neue Datei geladen, verh�lt sich die Baugruppe durch die definierten Beziehungen wie ein einzelner K�rper. Auf diesem Weg k�nnen in eine Baugruppe auch andere Baugruppen geladen und miteinander in Beziehung gestellt werden. Das macht ein schrittweises und �bersichtliches Zusammensetzen der Gesamtkonstruktion m�glich.  

\subsection{SolidEdge - Umsetzung} \label{subsec:Umsetzung}
Es soll nun ein Einblick in die Vorgehensweise bei der Konstruktion des Unterbaus mit SolidEdge gegeben werden. Abbildung \ref{fig:Unterbau} zeigt die Baugruppe \glqq Unterbau\grqq. Folgende Kriterien galt es dabei zu erf�llen: Die Baugruppe sollte aus mehreren K�rpern besteht, sodass ein einfaches Montieren der einzelnen Komponenten m�glich ist. Weiterhin sollte eine flexible Konstruktion entwickelt werden, um den Ballbot auf B�llen mit verschiedenen Radien testen zu k�nnen. Dies wurde durch eine Teleskopsystem erm�glicht, sodass die Motoren inklusive R�der radial nach au�en bzw. innen geschoben werden k�nnen. Dies ist in Abbildung \ref{fig:Motorenhalter} dargestellt. 
\begin{figure}[!htbp]%
	\centering
	{\includegraphics[scale=4.9]{Bilder/Michael/Traeger_Halterung_Explosion.jpg} }
	\caption{Teleskopsystem bestehend aus Tr�gerplatte(links) und Motorenhalter(rechts)}
	\label{fig:Motorenhalter}
\end{figure}

R�cksicht wurde weiterhin auf eine stabile Integration der Motoren in die Gesamtkonstruktion genommen. So soll bei den wirkenden Kr�ften und Momenten, Bewegungen der Unterbaukomponenten relativ zu einander vermieden werden. Integriert wurden die Motoren daher �ber vier M3 Zylinderkopfschrauben an den sogenannten Motorenhaltern, wie in Abbildung \ref{fig:antrieb} dargestellt. Die F�hrungsschiene des Motorenhalters  wird in eine F�hrungsnut der Tr�gerplatte passgenau eingef�hrt und verschraubt.


\begin{figure}[!htbp]%
	\centering
	{\includegraphics[scale=4.8]{Bilder/Michael/Motor_Mitnehmer_Halterung_Explosion.jpg} }
	\caption{Antriebsvorrichtung bestehend aus Mitnehmer (unten), Motor (mitte) und Motorenhalter (oben). }
	\label{fig:antrieb}
\end{figure}

Um die R�der, dargestellt in Abbildung \ref{fig:wheel} an die Motorwellen zu befestigen, sind spezielle Mitnehmer konstruiert worden, die in Abbildung \ref{fig:antrieb} dargestellt sind.

Wichtige Kriterien, die es hierbei zu erf�llen gab, waren eine spielfrei �bertragung der Momente und Kr�fte sowie eine m�glichst kurze Distanz zwischen den R�dern und den Motoren.

Nach einigen Tests hat sich herausgestellt, dass die gew�nschte Stabilit�t es Unterbaus nicht gegeben war. Die Konstruktion musste also noch verst�rkt werden. Um die n�tige Stabilit�t herzustellen, sind die Motorenhalter um eine Anflanschfl�che erweitert worden. So k�nnen die Motorenhalter �ber ein Y-f�rmiges Verbindungsst�ck, dargestellt in Abbildung \ref{fig:kreisring}, miteinander gekoppelt werden. Hierbei wurde eine flexible Anpassung an unterschiedlichen Ballradien ber�cksichtigt.
\begin{figure}[!htbp]%
	\centering
	{\includegraphics[scale=6.4]{Bilder/Michael/Wheel2_Double.jpg} }
	\caption{Modell eines omnidirektionalen Rades}
	\label{fig:wheel}
\end{figure}

\begin{figure}[!htbp]%
	\centering
	{\includegraphics[scale=4.8]{Bilder/Michael/Kreisring_Stabi.jpg} }
	\caption{Y-f�rmiges Verbindungsst�ck des Unterbaus.}
	\label{fig:kreisring}
\end{figure}

\section{Fertigung der Komponenten} \label{sec:Fertigung}
Neben dem Zugang zu einem professionellen CAD-Tool bestand auch Zugang zu einem 3D-Drucker der Firma Oktoprint. Dies erm�glichte eine kosteng�nstige und schnelle Fertigung der konstruierten Bauteile. Verwendete wurde dabei das Filament Polylactide (PLA). PLA ist ein sehr verbreiteter biokompatibler Kunstoff, der eine hohe Oberfl�chenh�rte, hohe Steifigkeit und eine hohe Zugfestigkeit bietet \cite{pla}. 

Der fertige Entwurf wurde schlie�lich als .stl exportiert und in das 3D-Druckprogramm CURA geladen. Mit CURA konnte anschlie�end die F�lldichte, die Qualit�t, Druckgeschwindigkeit und der Einsatz von St�tzhilfen f�r den jeweiligen Druck eingestellt werden. Die gew�nschten Genauigkeiten konnten mittels dieser Fertigungsweise eingehalten werden. So hat der Druck auch an kritischen Stellen wie der Einschubverbindung zwischen Tr�gerplatte und Motorenhalter �berzeugt. Es war nach Einschub und Verschraubung der Motorenhalter mit der Tr�gerplatte kein Spiel vorhanden. 
\chapter{Simulation}
Die Simulation des Ballbot's wurde mittels Gazebo realisiert und kann mittels eines globalen ROS launch files gestartet werden. Sie steht frei zur Verf�gung und der Programmcode kann �ber \cite{simulation_code} abgerufen werden. Zudem gibt es ein Youtube Video \cite{gazebo_anleitung} das zeigt, wie man die Simulation auf einem Ubuntu-Betriebssystem mit ROS-Kinetic und Gazebo7 testen kann.

\section{3D Simulatoren}
F�r eine 3D Simulation des Ballbot's bieten sich grunds�tzlich zwei 3D Simulatoren an: Gazebo und V-Rep. 
Die Unterschiede dieser beiden Simulatoren sind in Tabelle \ref{table:vrep_gazebo} aufgef�hrt.

\begin{table}
	\caption{Unterschiede zwischen den 3D Simulatoren V-REP und Gazebo.}
    \label{table:vrep_gazebo}
	\begin{tabular}{L{2cm}|L{6cm}|L{6cm}}
		Kriterium&Gazebo-Simulator&V-Rep-Simulator\\ \hline
		Lizenz&Open Source Programm&Kommerzielle und kostenlose Version\\ 
		ROS Integration&Gazebo ist der Standart Simulator von ROS. \newline Gazebo wird als ein ROS Node behandelt und kann daher sehr gut in ROS integriert werden.& V-REP hat keine direkte Anbindung an ROS. Es l�uft neben ROS in einem extra Terminal. \newline Jedoch existiert ein Plugin mit dem auf ROS Topics und Services zugegriffen werden kann.\\
		Plugins f�r Sensoren&Gazebo stellt bereits einige Plugins f�r Kameras, Laser Scanner etc. bereit. Diese k�nnen in einem xml file definiert werden.&V-REP stellt eine sehr benutzerfreundliche graphische Methode zur Verf�gung um ein Modell mit Sensoren auszustatten.\\
		CPU Auslastung&Gazebo lastet die Hardware sehr stark aus und ist bis zu 20\%  langsamer als V-REP.&V-REP hat im Gegensatz zu Gazebo eine konstante CPU Auslastung beim Zugriff auf ROS Nodes. \\
		Community&Gazebo hat eine rie�ige Community die sehr viele Plugins f�r neue Sensoren selbst entwickelt und zur Verf�gung stellt. Fragen gestellt werden.&V-REP ist nicht ganz so bekannt und hat lediglich 2570(01.2018) Forenmitglieder. Gazebo hat dagegen 3200.\\
	\end{tabular}
\end{table}


Auf dem Up-level Computer (dem UP-Board) des Ballbots soll ROS\footnote{Das Robot Operating System (ROS) ist eine Open Source Middelware. ROS hat eine rie�ige Community und erm�glicht es die verschiedenen Komponenten eines Roboters (Sensoren und Aktoren) geschickt miteinander zu verbinden. So ist es m�glich einen Roboter sehr elegant zu simulieren und mittels eins UP-Level Computers zu steueren.} laufen. Aus diesem Grund haben wir uns f�r den Gazebo Simulator entschieden, denn dieser ist der standart Simulator von ROS und daher besser integriert als V-REP.\cite[S. 1]{serena_vrep_vs_gz}

\section{Simulation von omnidirektionalen R�dern}
Bei der Simulation des Ballbot's in Gazebo stellt sich zun�chst die Frage wie man die omnidirektionalen R�der simulieren soll. Hierzu gibt es zwei M�glichkeiten: 

Die erste M�glichkeit besteht darin, das Omnidirektionale Rad ohne freidrehende kleine R�der zu simulieren. Um die freidrehenden kleinen R�der zu simulieren gibt man dem Rad zwei verschiedene Reibungskoeffizienten f�r die unterschiedlichen Reibungsrichtungen vor. Die Reibungskoeffizienten werden in gazebo mu1 und mu2 genannt. Der erste Reibungskoeffizient muss hierbei so gew�hlt werden, dass er der Reibung des Balls entspricht. Der zweite Reibungskoeffizient muss zu 0 gew�hlt werden, denn dieser simuliert die freidrehenden kleinen R�der des Omnidirektionalen Rades. Zus�tzlich muss man noch den fdir1 parameter von gazebo setzen. Dieser gibt an in welche Richtung des aktuellen Gelenks (Joints) der mu1 parameter zeigen soll. 

In Abbildung \ref{fig:simple_simulation} a) ist die einfachste Modellierungs-M�glichkeit eines Ballbot's dargestellt. In dem Ausschnitt in Abbildung \ref{fig:simple_simulation} b) sieht man, wie man die Reibungskoeffizienten Mu1 und Mu2 sowie den fdir1 Parameter f�r dieses Rad einstellen m�sste, um unendlich viele freie R�der zu simulieren. Leider hat diese einfache Modellierungsm�glichkeit beim Testen nicht funktioniert, da in Gazebo7 der fdir1 Parameter nicht richtig funktioniert (siehe ....).\footnote{Seit Gazebo8 sollte der fdir1 Parameter wieder richtig funktionieren. Dies wurde jedoch in dieser Arbeit nicht weiter betrachtet.} %https://bitbucket.org/osrf/gazebo/issues/463/ode-fdir1-parameter-broken )

\begin{figure}[!htbp]%
	\centering
	\subfloat[�bersicht des gesamten Aufbau's des Ballbots.]{{\includegraphics[width=5cm]{./Bilder/Markus/easy_model.png} }}%
	\qquad \qquad
	\subfloat[Simulation der freien R�der des Omnidirektionalen Rades.]{{\includegraphics[width=6cm]{./Bilder/Markus/simulation_easy.png} }}%
	\caption{Die einfachste Simulations M�glichkeit eines Ballbots in Gazebo7.}%
	\label{fig:simple_simulation}%
\end{figure}

Die zweite M�glichkeit ist sehr viel aufwendiger, denn sie besteht darin, das echte Omnidirektionale Rad mit allen kleinen R�dern zu simulieren. Hierf�r muss zun�chst das Omnidirektionale Rad ohne die freilaufenden kleine Subr�der in die Gazebo Simulation geladen werden. Anschlie�end m�ssen die Subr�der mit richtiger Orientierung und Position ebenfalls in die Simulation geladen werden. Das Omnidirektionale Rad sowie ein einzelnes Subrad wurde hierf�r zun�chst in Solid Edge konstruiert und anschlie�end als .stl exportiert. Diese .stl Dateinen werden dann mittels einer .xml Datei in die Gazebo Simulation geladen. 

F�r unsere finale Ballbot Simulation haben wir die zweite M�glichkeit benutzt, denn sie ist der Realit�t sehr viel N�her als die Modellierung von unendlich vielen kleinen Subr�dern. Nachteilig bei der Simulation aller 30 Subr�der ist jedoch, der deutlich gr��ere Berechnungsaufwand, der in unserem Falle die Simulation auf einen RealTime Faktor von 0.2 verlangsamt hat. Das hei�t die Simulation lief f�nf mal langsamer als sie in Echtzeit laufen w�rde.

\section{Simulations Aufbau}
Dieses Kapitel zeigt zun�chst exemplarisch die Kommunikation zwischen ROS und Gazebo. Anschlie�end wird n�her auf die verwendeten Plugins eingeganen.
Zudem wird auf dynamische Eigenschaften der Simulation wie etwa die Steifigkeit einzelner Elemente eingeganen. Zum Schluss wird noch auf zwei Visualisierungsprogramme(RVIZ und PlotJuggler)  eingegangen. 

\subsection{Kommunikation zwischen ROS und Gazebo}
Die Simulation mittels Gazebo wurde komplett in ROS aufgebaut. Sie besteht aus mehreren Nodes (Teilprogrammen) die alle durch ein globales ROS launch file gestartet werden. Beim Ausf�hren der Simulation sind sehr viele Teilprogramme(Nodes) aktiv die untereinander �ber sogenannte Topics Nachrichten austauschen. 
\begin{figure}[htbp]%
	{\includegraphics[scale=0.28]{./Bilder/Markus/rosgraph.png} }
	\caption{�bersicht aller Teilprogramme(Nodes) und deren Topics, die beim Starten der Simulation aktiv sind und Nachrichten austauschen. Hierbei sind die Nodes mit Ellipsen und die Topics mit Rechtecken gekennzeichnet. Das Bild wurde mit dem ROS Programm rqt\_graph erstellt.}
	\label{fig:ros_graph}
\end{figure}
Abbildung \ref{fig:ros_graph} zeigt den sogenannten ROS Graph der aktiven Nodes und deren Topics nach dem Starten des globalen ROS launch files. In der Mitte dieser Abbildung sieht man das den Node /ballbot/bb\_control des namspaces ballbot. Dies ist das Teilprogramm das die Regelung des simulierten Ballbots beinhaltet. Hierf�r liest(subscribt) es Nachrichten der Topics /ballbot/joints/joint\_states\footnote{Dieses Topic enth�lt eine Nachricht die die aktuellen Rad Drehmomente, Geschwindigkeiten [rad/sec] und deren absolute Positionen enth�lt. F�r die Regelung der Odometrie werden jedoch nur die Rad Drehgeschwindigkeiten verwendet. } und /ballbot/sensor/imu ein, berechnet damit die entsprechenden Drehmomente f�r die einzelnen R�der und ver�ffentlicht (published) die Nachrichten mit den berechneten Drehmomenten auf den Topics /ballbot/wheelx\_effort\_controller/command. Der Node /gazebo wiederum liest diese Drehmoment Befehle der einzelnen Omnidirektionalen R�der ein und dreht entsprechend in der sichtbaren Simulation die R�der. 

Es sei noch darauf hingewiesen, dass es m�glich ist Gazebo mit ROS zu synchronisieren. M�chte man zum Beispiel einen Regler implementieren, der eine sehr gro�e Update Frequenz (bzw. eine geringe Sample Time) aufwei�t, die Berechnung der Verst�rkungsfaktoren dieser Regelung jedoch l�nger dauert als die Sample Time, so muss die Regelung mit Gazebo synronisiert werden. Hierf�r gibt es die M�glichkeit, Gazebo pausiert zu starten und auch die ganze Zeit pausiert zu lassen. Ist nun die Berechnung der Regelung fertig, schickt man einen rosservice call an gazebo um die Simulation einen Schritt weiterlaufen zu lassen. Anschlie�end wird der n�chste Regelunswert berechnet. Bei dem simulierten Ballbot wurde eine Update Frequenz von 100Hz benutzt. Die Verst�rkungsfaktoren der 2D Regelung wurden jedoch mit mindestens 1000Hz berechnet. Daher musste Gazebo nicht mit dem entsprechendem ROS Node, der die Regelung darstellt (/ballbot\_controll) synronisiert werden. 


\subsection{Plugins der Simulation}
Die Sensoren des Ballbot's k�nnen in Gazebo durch Plugins simuliert werden. Abbildung \ref{fig:plugins_simulation} zeigt die simulierten Sensoren und deren Plugin Bibliotheken. Hierbei wurde die IMU mit einer update\_rate von 200Hz sowie mit einem Gau�schen Rausschen von 0.01 simuliert. Als Motoren Interface wurde das Joint Effort Interface verwendet, welches �ber das ros\_control Paket\footnote{Dieses Paket ist standardm��ig bei der ROS-Kinetic Version dabei. Weitere Informationen zu diesem Paket gibt es hier: http://wiki.ros.org/ros\_control .} zur Verf�gung steht. Die Motoren wurden dabei mit einem PID Regler simuliert. Dabei wurden die Verst�rkungsfaktoren zu P=100, I=0.01 und D=10 gew�hlt. F�r weitere Details zur Implementierung der Real-Sense Kamera sowie des LDS Laser Scanners sei auf den Programmcode\footnote{siehe: https://github.com/CesMak/bb/blob/master/src/ballbot/bb\_descrpition/urdf/bb\_double\_wheel.gazebo.xacro} verwiesen.

\begin{figure}[!htbp]%
	\centering
	{\includegraphics[scale=0.65]{./Bilder/Markus/simulation_plugins.png} }
	\caption{Der Simulierte Ballbot mit den simulierten Sensoren und deren Plugin Bibliotheken.}
	\label{fig:plugins_simulation}
\end{figure}
\subsection{Simulierte Dynamik-Eigenschaften}


%see here:http://ode.org/ode-latest-userguide.html
Nicht an dieser tabelle orientieren.
\begin{tabular}{l l l l}
	name(xacro)&description&value&sdf group\\
	mu1&is the Coulomb friction coefficient for the first friction direction&1.0&ode\\
	mu2& is the friction coefficient for the second friction direction (perpendicular to the first friction direction)&2.0&ode\\
	fdir1& 3-tuple specifying direction of mu1 in the collision local reference frame. fdir1 is the vector that defines the direction of mu1, which is the principal contact direction &0 0 0&ode\\
	kp&spring constant equivalents of a contact as a function of SurfaceParams::cfm and SurfaceParams::erp & &ode \\
	kd&spring damping constant equivalents of a contact as a function of SurfaceParams::cfm and SurfaceParams::erp.   & &ode \\
	cfm&Constraint Force Mixing parameter.& &ode \\
	erp&Error Reduction Parameter.& &ode \\
	min\_depth&Minimum depth before ERP takes effect.   & &ode \\
	max\_Vel& Maximum interpenetration error correction velocity.
	If set to 0, two objects interpenetrating each other will not be pushed apart.  & &ode \\
	slip1&Artificial contact slip in the primary friction direction  & &ode \\
	slip2&Artificial contact slip in the secondary friction dirction.& &ode \\
\end{tabular}

Ball so hart wie m�glich machen! in der simulation da wie in annahmen gesagt.

\subsection{Visualisierungsprogramme}
RVIZ, RQT-Multiplot


\chapter{Ausblick}
In den vorangegangenen Kapitel ist gezeigt worden, wie man ein Ballbot konzeptioniert, ihn modelliert, regelt, simuliert und baut. Es wurde weiterhin eine L�sung pr�sentiert, die zeigt wie man mit relativ einfachen L�sungsans�tzen zu einem akzeptablen Regelverhalten kommt. Dieses Regelverhalten weist jedoch Grenzen auf. In diesem Kapitel werden daher Verbesserungsm�glichkeiten genannt, die das Regelverhalten robuster gestalten sollen. Dabei gibt es mehrerer Punkte an den man Verbesserungen ansetzen kann: 

\begin{itemize}\itemsep-0.5\parsep
    \item \text{Ball:}\
        \ Es konnte zwar ein Ball gefunden werden, der einen guten Kompromiss(vgl. Kapitel \ref{sec:ball}) zwischen den gew�nschten Eigenschaften bietet, allerdings k�nnte dieser Kompromiss noch weiter verbessert werden. So kann zum Beispiel mit einer ma�genauen Aluminiumhohlkugel mit Gummibeschichtung die n�tige notwendige Steifigkeit und den Reibwert bereitstellen.
        
    \item \text{Filterung des Messdaten:}\
        \ Auch die Filterung hat einen gro�en Einfluss auf die Regelg�te. Im bestehenden Ballbot ist ein einfacher Mittelwertfilter verwendet worden. Dadurch konnte das Rauschen in einigen F�llen schon um den Faktor 2 reduziert werden, jedoch ist das Rauschen weiterhin im Regelverhalten zu sp�ren. Eine Verbesserung in diesem Verhalten k�nnte durch ein besseres Filter beispielsweise ein Kalmanfilter erzielt werden. Dies h�tte zudem den Vorteil, dass kein Zeitverzug entstehen w�rde.
        
    \item \text{Modellierung:}\
        \ Auch bei der Modellierung des Ballbots sind Vereinfachungen getroffen worden. Es hat sich gezeigt, dass die Vereinfachung zul�ssig sind und eine zweckm��ige Regelung implementiert werden kann. Allerdings werden damit Verkopplungen zwischen der xz- und yz-Ebene vernachl�ssigt. Diese sind zwar relativ klein, bei Ber�cksichtigung dieser Verkopplungen k�nnte jedoch sicherlich ein noch besseres Regelverhalten erzielt werden. Dies k�nnte durch eine 3D-Regelung erzielen werden. 
        
     \item \text{Motoren:}\
      \ Die im Projekt verwendeten Motoren weisen Drehzahlbegrenzungen auf. Sowohl Experimente als auch Simulation konnten zeigen, dass das System nicht wie vorgesehen stabilisiert werden kann. Durch die Verwendung anderer Motoren, die eine h�here Drehzahlbegrenzung aufweisen kann dieser Effekt umgangen werden.
\end{itemize}
% =================================================================================
% Anhang
% =================================================================================
\appendix % Damit wird der Anhang begonnen. Die Kapitel werden ab jetzt mit Buchstaben nummeriert

\chapter{Parameterliste} \label{ch:Parameterliste}
\begin{table*}[ht]
	\centering
		\begin{tabular}{|cccc|}
		\hline
		\textbf{Parameter} & \textbf{Variable} & \textbf{Wert} & \textbf{Quelle} \\ \hline
		\multicolumn{1}{|c}{Masse Gesamtaufbau (alles)} & \multicolumn{1}{c} {$--$} &  \multicolumn{1}{c} {1,731 kg} & \multicolumn{1}{c|} {Gemessen}  \\ \hline
		\multicolumn{1}{|c}{Masse Ball} & \multicolumn{1}{c} {$m_{S}$} &  \multicolumn{1}{c} {0,3280 kg} & \multicolumn{1}{c|} {Gemessen}  \\ \hline
		\multicolumn{1}{|c}{Masse Motor} & \multicolumn{1}{c} {$m_{M}$} &  \multicolumn{1}{c} {0,0820 kg} & \multicolumn{1}{c|} {Datenblatt}  \\ \hline
		\multicolumn{1}{|c}{Masse omnidirektionales Rad} & \multicolumn{1}{c} {$m_{OW}$} &  \multicolumn{1}{c} {0.0520 kg} & \multicolumn{1}{c|} {Gemessen}  \\ \hline
		\multicolumn{1}{|c}{Masse virtuelles Rad} & \multicolumn{1}{c} {$m_{W}$} &  \multicolumn{1}{c} {0,4020 kg} & \multicolumn{1}{c|} {Gemessen}  \\ \hline
		\multicolumn{1}{|c}{\begin{tabular}[c]{@{}c@{}}Masse Roboterk�rper\\ (mit Motoren/R�der)\end{tabular}}  & \multicolumn{1}{c} {$m_{B}$} & \multicolumn{1}{c} {1,603 kg} & \multicolumn{1}{c|} {Gemessen}\\ \hline
		\multicolumn{1}{|c}{\begin{tabular}[c]{@{}c@{}}Masse Roboterk�rper\\ (ohne Motoren/R�der)\end{tabular}}  & \multicolumn{1}{c} {$m_{B}$} & \multicolumn{1}{c} {1,2010 kg} & \multicolumn{1}{c|} {Gemessen}\\ \hline
		
		\multicolumn{1}{|c}{Radius Ball} & \multicolumn{1}{c} {$r_{S}$} &  \multicolumn{1}{c} {0,0800 m} & \multicolumn{1}{c|} {Datenblatt}  \\ \hline
		\multicolumn{1}{|c}{Radius virtuelles Rad} & \multicolumn{1}{c} {$r_{W}$} &  \multicolumn{1}{c} {0,0300 m} & \multicolumn{1}{c|} {Datenblatt}  \\ \hline
		\multicolumn{1}{|c}{Radius K�rper} & \multicolumn{1}{c} {$r_{B}$} &  \multicolumn{1}{c} {0,0703 m} & \multicolumn{1}{c|} {Gemessen}  \\ \hline
		\multicolumn{1}{|c}{H�he Massenschwerpunkt} & \multicolumn{1}{c} {$l$} &  \multicolumn{1}{c} {0.236 m} & \multicolumn{1}{c|} {SolidEdge}  \\ \hline
		\multicolumn{1}{|c}{H�he K�rper} & \multicolumn{1}{c} {$h$} &  \multicolumn{1}{c} {0.366 m} & \multicolumn{1}{c|} {SolidEdge}  \\ \hline
		
		
		\multicolumn{1}{|c}{Tr�gheitsmoment Ball} & \multicolumn{1}{c} {$I_{S}$} &  \multicolumn{1}{c} {0,0013 $kgm^{2}$} & \multicolumn{1}{c|} {Berechnet}  \\ \hline
		\multicolumn{1}{|c}{Tr�gheitsmoment Rotor} & \multicolumn{1}{c} {$I_{M}$} &  \multicolumn{1}{c} {3.8e-8 $kgm^{2}$} & \multicolumn{1}{c|} {Datenblatt}  \\ \hline
		\multicolumn{1}{|c}{Tr�gheitsmoment omnidirektionales Rad} & \multicolumn{1}{c} {$I_{OW}$} &  \multicolumn{1}{c} {2,34e-5 $kgm^{2}$} & \multicolumn{1}{c|} {Berechnet}  \\ \hline
		
		\multicolumn{1}{|c}{\begin{tabular}[c]{@{}c@{}}Tr�gheitsmoment der virtuellen R�der\\ ($yz$- und $xz$-Ebene)\end{tabular}}  & \multicolumn{1}{c} {$I_{W,yz, xz}$} & \multicolumn{1}{c} {0.00357 $kgm^{2}$} & \multicolumn{1}{c|} {Berechnet}\\ \hline
		
		\multicolumn{1}{|c}{\begin{tabular}[c]{@{}c@{}}Tr�gheitsmoment virtuelles Rad\\ ($xy$-Ebene)\end{tabular}}  & \multicolumn{1}{c} {$I_{W,xy}$} & \multicolumn{1}{c} {0.0143 $kgm^{2}$} & \multicolumn{1}{c|} {Berechnet}\\ \hline
		
		\multicolumn{1}{|c}{\begin{tabular}[c]{@{}c@{}}Tr�gheitsmoment K�rper\\ ($yz$-Ebene)\end{tabular}}  & \multicolumn{1}{c} {$I_{B,yz}$} & \multicolumn{1}{c} {0.0880 $kgm^{2}$} & \multicolumn{1}{c|} {SolidEdge}\\ \hline
		\multicolumn{1}{|c}{\begin{tabular}[c]{@{}c@{}}Tr�gheitsmoment K�rper\\ ($xz$-Ebene)\end{tabular}}  & \multicolumn{1}{c} {$I_{B,xz}$} & \multicolumn{1}{c} {0.0880 $kgm^{2}$} & \multicolumn{1}{c|} {SolidEdge}\\ \hline
		\multicolumn{1}{|c}{\begin{tabular}[c]{@{}c@{}}Tr�gheitsmoment K�rper\\ ($xy$-Ebene)\end{tabular}}  & \multicolumn{1}{c} {$I_{B,xy}$} & \multicolumn{1}{c} {0.0070 $kgm^{2}$} & \multicolumn{1}{c|} {SolidEdge}\\ \hline
		
		\multicolumn{1}{|c}{�bersetzungsverh�ltnis} & \multicolumn{1}{c} {$i$} &  \multicolumn{1}{c} {353,5} & \multicolumn{1}{c|} {Datenblatt}  \\ \hline
		\multicolumn{1}{|c}{Erdbeschleunigung} & \multicolumn{1}{c} {$g$} &  \multicolumn{1}{c} {9,81 $\frac{m}{s^{2}}$} & \multicolumn{1}{c|} {Datenblatt}  \\ \hline
		\end{tabular}
\end{table*}

\chapter{Bestands-Liste}
\begin{table}[H]
	\centering
	\caption{Screws:}
	\label{screws}
	\begin{tabular}{|c|c|c|c|}
		\hline
		{\textbf{Type}} & {\textbf{Size}}                                             & { \textbf{Amount}} & { \textbf{Place}} \\ \hline
		Cylinderhead screw  & M3 x 11mm                                                       & 8                     & Motor mounts         \\ \hline
		Cylinderhead screw  & M2,5 x 22mm                                                     & 16                    & Motor plate          \\ \hline
		Cylinderhead screw  & M2 x 6 mm                                                       & 18                    & Wheel shaft          \\ \hline
		Cylinderhead screw  & \begin{tabular}[c]{@{}c@{}}M2,5  x 36 mm\\ (38 mm)\end{tabular} & 5                     & Wheel shaft cover    \\ \hline
		Cylinderhead screw  & \begin{tabular}[c]{@{}c@{}}M3 x 20 mm\\ (21mm)\end{tabular}     & 4                     & Layer mounting       \\ \hline
		Nut                 & M2                                                              & 5                     & Layer mounting       \\ \hline
		Cylinderhead screw  & \begin{tabular}[c]{@{}c@{}}M2,5 x 22mm\\ (23mm)\end{tabular}    & 4                     & Layer mounting       \\ \hline
	\end{tabular}
\end{table}

\begin{table}[H]
	\centering
	\caption{�bersicht �ber die verwendeten Bauteile}
	\label{bauteile1}
\begin{tabular}{l l l l l}
	Item & \# & W.[g] & Weblink & Bild\\ \hline
	OpenCR Board (Controlling the motors, IMU)&1&60&\mylink{https://github.com/ROBOTIS-GIT/OpenCR/wiki/Hardware_Specification\#specification}{github\_wiki} 
	&\includegraphics[width=0.1\textwidth]{./Bilder/Markus/img/opencr.png}  \\
		
	UpBoard (Main PC)&1 &96 & \mylink{https://up-shop.org/up-boards/44-up-board-4gb-ram-64-gb-emmc.html}{\EUR{127}}
	&\includegraphics[width=0.1\textwidth]{./Bilder/Markus/img/upboard.jpg} \\
		
	Intel RealSense R200&1& 9.4& \mylink{https://www.intel.de/content/www/de/de/support/articles/000023534/emerging-technologies/intel-realsense-technology.html}{datasheet, \EUR{84.15}}&
	\includegraphics[width=0.1\textwidth]{./Bilder/Markus/img/r200.jpg} \\
\end{tabular}
\end{table}

\begin{table}[]
	\centering
	\caption{�bersicht �ber die verwendeten Bauteile}
	\label{bauteile2}
\begin{tabular}{l l l l l}
	Item & \# & W.[g] & Weblink & Bild\\ \hline
			
			Laser Distance Sensor&1 &124 &\mylink{https://wiki.ros.org/hls_lfcd_lds_driver?action=AttachFile\&do=view\&target=LDS_Basic_Specification.pdf}{specs, \EUR{100}} & 
			\includegraphics[width=0.1\textwidth]{./Bilder/Markus/img/lasersensor.png}\\
			
			Battery: LI-PO 11.1 1800mAh LB-12 19&1&132 &\mylink{https://nodna.de/Robotis-LIPO-111V-Akkupack-1800mAh-LBS-012}{\EUR{44.90}} &
			\includegraphics[width=0.1\textwidth]{./Bilder/Markus/img/battery.png} \\
			
			Turtlebot3 Layers(125cmx125cm)&4& & & \\
			
			XM430-W350-R Dynamixel (Motors)&3&82 &\mylink{http://support.robotis.com/en/product/actuator/dynamixel_x/xm_series/xm430-w350.htm}{robotis,\EUR{250}} &
			\includegraphics[width=0.1\textwidth]{./Bilder/Markus/img/dynamixel.png}\\ 
			
			Ball(alum., dia.: 140mm, material thickness 2.5mm)&1&400&\mylink{http://www.ball-tech.de/Hohlkugeln/Aluminium/}{ball-tech gmbh,\EUR{40}. } & \includegraphics[width=0.1\textwidth]{./Bilder/Markus/img/ball.png}\\
			
			Omni wheels(dia: 60mm, thickness:25mm)&3&51.46 &\mylink{http://krause-robotics.de/xtshop/Antriebe/Raeder/Allseitenraeder/Allseitenraeder-60-mm:::99_100_106_114.html}{\EUR{10.38}}&\includegraphics[width=0.1\textwidth]{./Bilder/Markus/img/wheel.jpg}   \\
			
			Kreisring (PLA, 3D printeted)&1&28 & & 
			\includegraphics[width=0.1\textwidth]{./Bilder/Markus/img/kreisring.png}\\
			
			Halterung (PLA, 3D printeted)&3&18 & & 
			\includegraphics[width=0.1\textwidth]{./Bilder/Markus/img/halterung.png} \\
			
			Mitnehmer (PLA, 3D printeted)&3&8 & &
			\includegraphics[height=0.06\textwidth]{./Bilder/Markus/img/mitnehmer.png}  \\
			
			Plain washer (Beilagscheibe),(PLA, 3D printeted)&3&0.45 & &
			\includegraphics[height=0.06\textwidth]{./Bilder/Markus/img/beilagscheibe.png}  \\
			
			Omni double wheels(dia: 56mm, thickness:25mm)&3&62 &{\EUR{15}} &  	\includegraphics[width=0.05\textwidth]{./Bilder/Markus/img/wheel_double.jpg}\\
	
	Mitnehmer double wheels&3&7 & &  	\includegraphics[width=0.05\textwidth]{./Bilder/Markus/img/mitnehmer_double.png}   \\
	
	Ball(alum., dia.: 140mm, material thickness 2.5mm)&1&400&\mylink{http://www.ball-tech.de/Hohlkugeln/Aluminium/}{ball-tech gmbh,\EUR{40}. } & \includegraphics[width=0.1\textwidth]{./Bilder/Markus/img/white_ball.png}\\
	
	Ball(gummi, diameter: 150mm)&1&326&link and cost&\includegraphics[width=0.1\textwidth]{./Bilder/Markus/img/yellow_ball.png}\\	
\end{tabular}
\end{table}

Die Gesamtkosten der verwendeten Bauteile belief sich auf ca. \EUR{1200}.

\chapter{ROS-Graph Simulation} \label{fig:rosgraph}
\begin{figure}[htbp]%
	{\includegraphics[scale=0.25, angle=90]{./Bilder/Markus/rosgraph.png} }
	\caption{�bersicht aller Teilprogramme(Nodes) und deren Topics, die beim Starten der Simulation aktiv sind und Nachrichten austauschen. Hierbei sind die Nodes mit Ellipsen und die Topics mit Rechtecken gekennzeichnet. Das Bild wurde mit dem ROS Programm rqt\_graph erstellt.}
\end{figure}

\chapter{Simulink Simulationsaufbau} \label{ch:gesamtbild}
\begin{figure}[htbp]%
	{\includegraphics[scale=0.8, angle=90]{./Bilder/Florian/Komplett_System_MATLAB_SIMULINK.PNG} }
	\caption{�bersicht des Simulink Simulationsaufbaus}
\end{figure}




% =================================================================================


%% =================================================================================
%% Abbildungsverzeichnis
%% =================================================================================
%\cleardoublepage
%\phantomsection					% F�r Aufnahme ins Inhaltsverzeichnis
%\addcontentsline{toc}{chapter}{\listfigurename}	% In Inhaltsverzeichnis von
%												% Dokument und pdf aufnehmen
%\listoffigures
%% =================================================================================
%
%% =================================================================================
%% Tabellenverzeichnis
%% =================================================================================
%\cleardoublepage
%\phantomsection					% F�r Aufnahme ins Inhaltsverzeichnis
%\addcontentsline{toc}{chapter}{\listtablename}	% In Inhaltsverzeichnis von
%												% Dokument und pdf aufnehmen
%\listoftables
%% =================================================================================

% =================================================================================
% Literaturverzeichnis
% =================================================================================
\cleardoublepage
\phantomsection					% F�r Aufnahme ins Inhaltsverzeichnis
\addcontentsline{toc}{chapter}{\bibname}	% In Inhaltsverzeichnis von
											% Dokument und pdf aufnehmen
%\bibliographystyle{gerabbrv}	% Verweise nummeriert in eckigen Klammern, alphabetisch sortiert
\bibliographystyle{gerunsrt}	% Verweise nummeriert in eckigen Klammern, nach Erscheinung sortiert


\bibliography{./bib/literature}	% Literaturverzeichnis einf�gen, mit Angabe der
								% Bibtex-Datei
% =================================================================================
\end{document}
