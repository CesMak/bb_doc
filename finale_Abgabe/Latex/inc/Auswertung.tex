\chapter{Zusammenfassung}
Diese Arbeit behandelt die Implementierung eines balancierenden Ballbot basierend auf einer gegebenen Hardware. Im Verlauf der Arbeit ist dazu die gegebene Hardware durch eine selbst hergestellte Hardware erg�nzt, mehrere Simulation mit Gazebo sowie MATLAB/Simulink zur Validierung der erzeugten Modelle durchgef�hrt, sowie eine Regelung f�r das reale System entworfen worden.

Der zeitliche Ablauf der Arbeit gliedert sich wie folgt. Zu Beginn wurde ein Konzept, basierend auf bereits bestehenden Ballbot-Projekten entworfen. Dieses hat sich dann als eine zylinderf�rmiger K�rper ergeben, der �ber drei, um 120 Grad verdrehte, omnidirektionale R�der auf einem Ball balancieren soll. Im Anschluss wurde mit dem Entwurf des gesamten Ballbots in SolidEdge begonnen. Dabei wurde die Konstruktion zur Aufnahme der Antriebe entwickelt und mit der basierenden Hardware verkn�pft. Neben einer genauen Entwicklung des Aufbaus zur Aufnahme der Antriebe, hatten die Arbeiten mit SolidEdge weiterhin zum Ziel ein detailgetreues Modell der Realit�t in der Simulationsumgebung Gazebo simulieren zu k�nnen. Mittels dieser Simulationsumgebung konnte so im Anschluss an die Konstruktionsarbeiten und des Reglerentwurfs, das Gesamtsystem unter labor- als auch realit�tsnahen Bedingungen ausgiebig getestet werden. Dies hat sich f�r die sp�teren Test am realen Ballbot als sehr n�tzlich erwiesen, da man das Verhalten bereits untersuchen und kennenlernen konnte. So konnte gezeigt werden, dass die verwendeten Motoren f�r die Ballbot Anwendung nur bedingt geeignet sind, da sie nur ein sehr kleines Drehzahlband besitzen. 

Zeitgleich zum Einlesen in die Simulationsumgebung ist der Regler mit MATLAB/Simulink entworfen worden. F�r das Gesamtsystem wurde ein LQR-Zustandsregler, unter der Annahme das reale System durch drei unabh�ngige planare Ebenen zu approximieren, entworfen. Durch eine mathematische Abbildung des realen Systems, implementiert in Simulink, wurde der entworfene Regler in seiner Wirkungsweise �berpr�ft. So konnte gezeigt werden, dass der Regler in der Lage ist den Ballbot um den betrachteten Arbeitspunkt zu stabilisieren. In Simulink ber�cksichtigt wurden dabei real auftretende St�rungen wie Schlupf und Totzeiten. 

Nach diesem Schritt wurde die Regelung auf dem realen System realisiert. Dazu wurde ein Programm auf einem Mikrocontroller erstellt, dass die Sensorik ausliest, diese Werte durch das in Kapitel \ref{ch:Modellierung} gewonnene Regelgesetz weiterverarbeitet und die Stellgr��e schlie�lich auf die Antriebe gibt. Beim Einlesen der Sensorwerte wurde weiterhin zur Rauschreduzierung ein Filter implementiert. Die Funktionsweise des so implementierten realen Gesamtsystems konnte im Anschluss auf einer d�nnen Schaumstoffmatte best�tigt werden.