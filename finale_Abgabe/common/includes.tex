% Dieses File dient zum einbinden wichtiger und n�tzlicher Pakete.
% Nicht alle Pakete m�ssen verwendet werden.
%
\usepackage{t1enc}			% evtl. dc-Fonts
\usepackage[T1]{fontenc}	% F�r Silbentrennung bei W�rten mit Sonderzeichen (z.B. Umlaute)
\usepackage[latin1]{inputenc}
							% Um Sonderzeichen (�, �, �, ...) direkt eingeben zu k�nnen
\usepackage[english,ngerman]{babel}
							% F�r Sprachenspezifisches
							% ngerman ist schon als globale Option definiert

%\usepackage{helvet}			% Helvetica als Standard-Sans-Schriftart
\usepackage[stable]{footmisc}
\usepackage{booktabs}


\usepackage{graphicx}		% zum Einbinden von Postscript
\usepackage{psfrag}			% Beschriftung der Bilder
\usepackage{amsmath}		% Mehr mathematischen Formelsatz
%\usepackage{amssymb}		% Mehr mathematische Symbole
%\usepackage{amsthm}

\usepackage{float}			% F�r Parameter [H] bei Flie�objekten

\usepackage{epsfig}			% Um eps-Bilder einzubinden
\usepackage{scrhack}    % Um Warnung "float@addtolists detected" zu unterdr�cken 
\usepackage{subfig}			% F�r Unterabbildungen
\usepackage{ltxtable} 		% Vereinigt TabularX und Longtable
\usepackage{longtable}
\usepackage{rotating}		% Zum Drehen von Objekten
\usepackage{bibgerm}		% F�r deutsche Literaturverwaltung
%\usepackage{wrapfig}		% F�r kleine Bilder am Rand
%\usepackage{floatflt}		% Alternative zu wrapfig
%\usepackage[hang]{caption}	% Damit mehrzeilige Bildunterschriften gut aussehen
\usepackage{upgreek}		% F�r nicht-kursive kleine griechischen Buchstaben

\usepackage{multirow}		% F�r mehrzeilige Felder in Tabellen

\usepackage{textcomp}		% F�r Sonderzeichen im normalen Text
							% (offensichtlich in tudreport schon eingebunden)
\usepackage[ngerman]{varioref}		% F�r vref
\usepackage{color}			% F�r farbigen Text
\usepackage{placeins}		% F�r \FloatBarrier
\usepackage{xspace}
\usepackage{icomma}			% Damit nach Dezimalkommas kein Abstand eingef�gt wird
							% (in math-Umgebungen)

\usepackage{cancel}			% Zum Wegstreichen von Gleichungstermen

\usepackage{array}			% F�r Zellentyp "m{}" in tabular-Umgebungen (Vertikal zentriert)

\usepackage{listings}		% Um formatierten Quellcode einzubinden
\usepackage{moreverb}		% F�r Umgebung "`verbatimtab"' (Verbatim mit Tabs)
\renewcommand{\verbatimtabsize}{4\relax}	% Standardtabweite in "`verbatimtab"'

											% ist 4 Zeichen


% Das Packet hyperref immer als letztes einbinden (nur bookmark darf danach kommen)!
%\usepackage[ps2pdf, colorlinks=false, pdfborder={0 0 0}]{hyperref}
%\usepackage[ps2pdf]{hyperref}	% F�r Verlinkungen im erzeugten pdf
\usepackage{hyperref}	% F�r Verlinkungen im erzeugten pdf
\usepackage{bookmark}
\usepackage{url}

% Markus includes:
\usepackage{array}
\newcolumntype{L}[1]{>{\raggedright\let\newline\\\arraybackslash\hspace{0pt}}m{#1}}
\newcolumntype{C}[1]{>{\centering\let\newline\\\arraybackslash\hspace{0pt}}m{#1}}
\newcolumntype{R}[1]{>{\raggedleft\let\newline\\\arraybackslash\hspace{0pt}}m{#1}}
\usepackage{footnote}
\makesavenoteenv{tabular} % footnotes in tabular env.

% Florian includes:
\usepackage{tikz}
%\usepackage{etoolbox}
%\makeatletter
%\AtBeginDocument{
%\patchcmd{\use@@tikzlibrary}{\global}{}{}{}
%}
%\makeatother
\usepackage{pgfplots} 
\usepackage{pgfgantt}
\usepackage{pdflscape}
\pgfplotsset{compat=newest} 
\pgfplotsset{plot coordinates/math parser=false}
\newlength\figH
\newlength\figW
\setlength{\figH}{7cm}
\setlength{\figW}{14cm}