\section*{Aufgabenstellung}
Im Rahmen dieses Projektseminars soll basierend auf am Fachgebiet SIM (Fachbereich Informatik) vorhandenen Komponenten ein Ballbot-Roboter nach dem in \cite{ETHZ} beschriebenen Beispiel aufgebaut und in Betrieb genommen werden.
\begin{itemize}
	\item Konstruktion der notwendigen mechanischen Baugruppen
	\item Aufbau des Roboters
	\item Inbetriebnahme von Aktorik und Sensorik �ber On-Board-Rechner
	\item Identifikation der Systemparameter sowie
	\item Modellierung und Regelung
\end{itemize}
zu bearbeiten.

Ziel der Arbeit ist es, einen funktionsf�higen Roboter zu haben, auf dem ein einfaches Balancieren auf der Stelle implementiert ist. Dabei kann die Regelung durchaus von bestehenden Arbeiten �bernommen werden. Unabh�ngig von den letztendlich implementierten Regelkonzepten wird jedoch eine ausf�hrliche Recherche zu den Regelkonzepten aus der Literatur gefordert.


\vspace{0.5cm}
\begin{tabular}{ll}
Beginn: & \SADABegin \\
Ende:   & \SADAAbgabe \\
Pr�sentation:& \SADASeminar
\end{tabular}

\vspace{1cm}

\begin{tabular}{ll}
\rule{7cm}{0.4pt} \hspace{1cm} & \rule{7cm}{0.4pt} \\
\SADAProf & \SADABetreuer\\
 &\SADABetreuerII\\
 &\SADABetreuerIII
\end{tabular}

\vfill
{\renewcommand{\baselinestretch}{1} % f�r diesen Abschnitt einfacher Zeilenabstand
\normalsize % anwenden des Zeilenabstandes
\begin{minipage}{0.8\textwidth}
	Technische Universit�t Darmstadt\\
	\SADAinstitut\\[0.5cm]
%
	Landgraf-Georg-Stra�e 4\\
	64283 Darmstadt\\
	Telefon \SADAtel\\
	\SADAwebsite
\end{minipage}
\begin{minipage}{0.2\textwidth}
\flushright  % rechtsb�ndig
\ \\[2.7cm]
\SADAlogo\;
\end{minipage}}

