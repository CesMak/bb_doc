\chapter{Ausblick}
In den vorangegangenen Kapitel ist gezeigt worden, wie man ein Ballbot konzeptioniert, ihn modelliert, regelt, simuliert und baut. Es wurde weiterhin eine L�sung pr�sentiert, die zeigt wie man mit relativ einfachen L�sungsans�tzen zu einem akzeptablen Regelverhalten kommt. Dieses Regelverhalten weist jedoch Grenzen auf. In diesem Kapitel werden daher Verbesserungsm�glichkeiten genannt, die das Regelverhalten robuster gestalten sollen. Dabei gibt es mehrerer Punkte an den man Verbesserungen ansetzen kann: 

\begin{itemize}\itemsep-0.5\parsep
    \item \text{Ball:}\
        \ Es konnte zwar ein Ball gefunden werden, der einen guten Kompromiss(vgl. Kapitel \ref{sec:ball}) zwischen den gew�nschten Eigenschaften bietet, allerdings k�nnte dieser Kompromiss noch weiter verbessert werden. So kann zum Beispiel mit einer ma�genauen Aluminiumhohlkugel mit Gummibeschichtung die n�tige notwendige Steifigkeit und den Reibwert bereitstellen.
        
    \item \text{Filterung des Messdaten:}\
        \ Auch die Filterung hat einen gro�en Einfluss auf die Regelg�te. Im bestehenden Ballbot ist ein einfacher Mittelwertfilter verwendet worden. Dadurch konnte das Rauschen in einigen F�llen schon um den Faktor 2 reduziert werden, jedoch ist das Rauschen weiterhin im Regelverhalten zu sp�ren. Eine Verbesserung in diesem Verhalten k�nnte durch ein besseres Filter beispielsweise ein Kalmanfilter erzielt werden. Dies h�tte zudem den Vorteil, dass kein Zeitverzug entstehen w�rde.
        
    \item \text{Modellierung:}\
        \ Auch bei der Modellierung des Ballbots sind Vereinfachungen getroffen worden. Es hat sich gezeigt, dass die Vereinfachung zul�ssig sind und eine zweckm��ige Regelung implementiert werden kann. Allerdings werden damit Verkopplungen zwischen der xz- und yz-Ebene vernachl�ssigt. Diese sind zwar relativ klein, bei Ber�cksichtigung dieser Verkopplungen k�nnte jedoch sicherlich ein noch besseres Regelverhalten erzielt werden. Dies k�nnte durch eine 3D-Regelung erzielen werden. 
        
     \item \text{Motoren:}\
      \ Die im Projekt verwendeten Motoren weisen Drehzahlbegrenzungen auf. Sowohl Experimente als auch Simulation konnten zeigen, dass das System nicht wie vorgesehen stabilisiert werden kann. Durch die Verwendung anderer Motoren, die eine h�here Drehzahlbegrenzung aufweisen kann dieser Effekt umgangen werden.
\end{itemize}