\selectlanguage{ngerman}
\maketitle

% Die Farbe der Identit�tsleiste wird auf Grau umgestellt, damit nicht alle Seiten
% farbig gedruckt werden m�ssen
\ifTUDdesign
	\ifOnlyColorFront	% ggf. nachfolgende Balken andere Farbe zuweisen
		\makeatletter 	% ben�tigt, um die @-Befehle auszuf�hren
    \renewcommand{\@TUD@largerulecolor}{\color{tud0b}}% am besten gleiche Farbe wie in der ersten Zeile und die Zahl durch die 0 ersetzen, dann hat das Grau die richtige Intensit�t
    \makeatother
	\fi
\fi


\pagenumbering{roman}	% Bis zum Hauptteil werden r�mische Seitenzahlen verwendet

% =================================================================================
% Spezielle Seiten f�r studentische Arbeiten
% =================================================================================
\cleardoublepage
\section*{Aufgabenstellung}
F�r schriftliche Arbeiten (Pro-/Projektseminar, Studien-, Bachelor-, Master-, Diplomarbeiten, \etc) soll Studierenden ein \LaTeX-Dokument zur Verf�gung gestellt werden, das die Vorgaben aus den \emph{Richtlinien zur Anfertigung von Studien- und Diplomarbeiten}~\cite{Richtlinien} umsetzt. Die Dokumentation soll die Funktionen des Dokumentes beschreiben und Hinweise zu ihrer Anwendung geben.

Grundlage ist die \verb|tudreport|-Klasse. Die damit erstellten Arbeiten m�ssen sowohl zum Ausdrucken geeignet sein als auch f�r die Bildschirmdarstellung und die elektronische Archivierung als PDF-Datei.

\vspace{0.5cm}
\begin{tabular}{ll}
Beginn: & \SADABegin \\
Ende:   & \SADAAbgabe \\
Seminar:& \SADASeminar
\end{tabular}

\vspace{1cm}

\begin{tabular}{ll}
\rule{7cm}{0.4pt} \hspace{1cm} & \rule{7cm}{0.4pt} \\
\SADAProf & \SADABetreuer\\
 &\SADABetreuerII\\
 &\SADABetreuerIII
\end{tabular}

\vfill
{\renewcommand{\baselinestretch}{1} % f�r diesen Abschnitt einfacher Zeilenabstand
\normalsize % anwenden des Zeilenabstandes
\begin{minipage}{0.8\textwidth}
	Technische Universit�t Darmstadt\\
	\SADAinstitut\\[0.5cm]
%
	Landgraf-Georg-Stra�e 4\\
	64283 Darmstadt\\
	Telefon \SADAtel\\
	\SADAwebsite
\end{minipage}
\begin{minipage}{0.2\textwidth}
\flushright  % rechtsb�ndig
\ \\[2.7cm]
\SADAlogo\;
\end{minipage}}



\cleardoublepage
\ \\[3cm]	% Diese Zeile erzeugt einen Abstand von 4cm zur ersten Zeile, die nur ein Leerzeichen
			% enth�lt

\ifx\SADAVarianteErklaerung\ETIT
	\section*{Erkl�rung}
	\noindent
	Hiermit versichere ich, dass ich die vorliegende Arbeit ohne Hilfe Dritter und nur mit den angegebenen Quellen und Hilfsmitteln angefertigt habe. Alle Stellen, die aus den Quellen entnommen wurden, sind als solche kenntlich gemacht. Diese Arbeit hat in gleicher oder �hnlicher Form noch keiner Pr�fungsbeh�rde vorgelegen.\vspace*{20mm} \\
	\noindent
	\begin{tabular}{ll}
		\SADAStadt, den \SADAAbgabe	\hspace{1cm}	& \rule{0.4\textwidth}{0.4pt}\\
		& Florian M�ller\\ & \\ &\rule{0.4\textwidth}{0.4pt} \\ & Markus Lamprecht\\ & \\ & \rule{0.4\textwidth}{0.4pt} \\ & Michael Suffel
	\end{tabular}
	


\else\ifx\SADAVarianteErklaerung\MBDA
	\section*{Erkl�rungen}
	\noindent
	Hiermit erkl�re ich an Eides statt, dass ich die vorliegende \SADATyp\ mit dem Titel\ \glqq\SADATitel\grqq\ selb\-st�ndig und ohne fremde Hilfe verfasst, andere als die angegebenen Quellen und Hilfsmittel nicht benutzt und die aus anderen	Quellen entnommenen Stellen als solche gekennzeichnet habe.\\
	Diese Arbeit hat in gleicher oder �hnlicher Form noch keiner Pr�fungsbeh�rde vorgelegen.\vspace*{20mm} \\
	\noindent
	\begin{tabular}{ll}
		\SADAStadt, den \SADAAbgabe	\hspace{1cm}	& \rule{0.4\textwidth}{0.4pt}\\
										& \SADAAutor
	\end{tabular}
	
	
	\vspace{40mm}
	\noindent
	Ich bin damit einverstanden, dass die TU Darmstadt das Urheberrecht an meiner \SADATyp\ zu wissenschaftlichen Zwecken nutzen kann.\vspace*{20mm} \\
	\noindent
	\begin{tabular}{ll}
		\SADAStadt, den \SADAAbgabe	\hspace{1cm}	& \rule{0.4\textwidth}{0.4pt}\\
										& \SADAAutor
	\end{tabular}

	{\huge Hier fehlt noch was!}
	
	
\else\ifx\SADAVarianteErklaerung\MBSA
	\section*{Erkl�rungen}
	\noindent
	Hiermit erkl�re ich an Eides statt, dass ich die vorliegende \SADATyp\ mit dem Titel\ \glqq\SADATitel\grqq\ selb\-st�ndig und ohne fremde Hilfe verfasst, andere als die angegebenen Quellen und Hilfsmittel nicht benutzt und die aus anderen	Quellen entnommenen Stellen als solche gekennzeichnet habe.\\
	Diese Arbeit hat in gleicher oder �hnlicher Form noch keiner Pr�fungsbeh�rde vorgelegen.\vspace*{20mm} \\
	\noindent
	\begin{tabular}{ll}
		\SADAStadt, den \SADAAbgabe	\hspace{1cm}	& \rule{0.4\textwidth}{0.4pt}\\
										& \SADAAutor
	\end{tabular}
	
	
	\vspace{40mm}
	\noindent
	Ich bin damit einverstanden, dass die TU Darmstadt das Urheberrecht an meiner \SADATyp\ zu wissenschaftlichen Zwecken nutzen kann.\vspace*{20mm} \\
	\noindent
	\begin{tabular}{ll}
		\SADAStadt, den \SADAAbgabe	\hspace{1cm}	& \rule{0.4\textwidth}{0.4pt}\\
										& \SADAAutor
	\end{tabular}


\else
	{\huge Unbekannte Variante der Erkl�rung!}

\fi\fi\fi






\clearpage
\section*{Kurzfassung}
Inhalt dieser Arbeit ist der mechanische Aufbau, die Modellierung, der Reglerentwurf und die Simulation eines Ballbots. Aufgebaut wurde der Ballbot mittels Komponenten eines Turtlebot3 Roboter's, der um einen selbst entwickelten Unterbau erg�nzt wurde. Anschlie�end erfolgte die mathematische Modellbildung und eine linear-quadratische Reglerauslegung(LQR) in MATLAB/Simulink. Bei der Modellbildung wurde das reale dreidimensionale System durch zwei unabh�ngige planare Ebenen ($xz$ und $yz$) approximiert. Der entworfene Regler wurde schie�lich mit dem 3D-Simulator Gazebo validiert. 

Es konnte gezeigt werden, dass sich das System mittels der entworfenen Regelung in der $xz$- und $yz$-Ebene stabilisieren l�sst. Essentiell f�r das Balancieren des Ballbot's ist eine optimale Abstimmung des Roboters auf den Ball. Es sollte darauf geachtet werden dass:
\begin{itemize}
	\item Der Reibkoeffizient im Kontaktpunkt zwischen Boden und Ball, sowie zwischen Ball und omnidirekionalem Rad muss m�glichst gro� sein.
	\item Die Motoren eine hohe maximale Drehzahl($>4.5 rad/sec$) aufweisen.
	\item Die Abtastrate des Systems m�glichst hoch($>130 Hz$) ist.
\end{itemize}

Weiterhin kann das System durch eine genauere Modellierung (3D Modellierung) sowie der Ber�cksichtigung der Ball Odometrie stabilisiert werden.

\textbf{Schl�sselw�rter:} Ballbot, Omnidirektionale R�der, Gazebo, Matlab, Arduino


\newpage
\selectlanguage{english}
\section*{Abstract}
Goal of this ressearch project was the construction, modelling, 
controller implementation and the simulation of a Ballbot. This Ballbot was built by using components of a Turtlebot3 Robot. Additionally a specific designed strucutre was created to attach the omnidirectional wheels to the ballbot. After that MATLAB/Simulink was used to design a linear-quadratic Regulator(LQR). The Modelling was done by approximating the real three dimensional System with two planes (xz and yz). At the end the Ballbot was validated with the 3D-Simulator Gazebo.

This project revealed, that the system could be stabilized around the xz- and the yz-plane. Essential for the balancing behaviour of the ballbot was an optimal adaption of the Robot to the ball. Thereby it should be considered:
\begin{itemize}
	\item The friction coefficient between the ball and the surface and between the ball and the omnidirectional wheel should be as high as possible.
	\item The maximum revolution speed of the motors should be as high as possible ($>4.5 rad/sec$).
	\item The Sampling Frequenzy of the System should be higher than $130 Hz$.
\end{itemize}

In order to improve the Stabilization of the System the 2D-Modelling can be replaced by a more exact 3D-Modelling. Additionally the Ball's Odometry can be included in the design of the controller.

\textbf{Keywords:} Ballbot, Omnidirectional Wheels, Gazebo, Matlab, Arduino 
\selectlanguage{ngerman} 
% =================================================================================

% =================================================================================
% Inhaltsverzeichnis
% =================================================================================
\cleardoublepage	% Auf einer leeren rechten Seite beginnen
\phantomsection		% Diese und die n�chste Zeile dient dazu, f�r das Inhalts-
					% verzeichnis einen Eintrag in das pdf-Inhaltsverzeichnis,
					% aber nicht in das normale Verzeichnis zu erzeugen.
\pdfbookmark[0]{\contentsname}{pdf:toc}	
\tableofcontents	% Inhaltsverzeichnis einf�gen
\clearpage	% Sonst kommt nichts mehr auf die Seite
% =================================================================================


% =================================================================================
% Symbole und Abk�rzungen
% =================================================================================
% Nach dem Inhaltsverzeichnis kommt ein Verzeichnis der h�ufig verwendeten
% Symbole und Abk�rzungen. Dazu kann man das Paket 'nomencl' verwenden, oder man
% erstellt es von Hand.
\chapter*{Symbole und Abk�rzungen}
\addcontentsline{toc}{chapter}{Symbole und Abk�rzungen} % erzeugt einen Eintrag im Inhaltsverzeichnis
%
\paragraph*{Lateinische Symbole und Formelzeichen}
\begin{tabularx}{\textwidth}{@{}l@{\qquad}X@{\quad}p{18mm}}
	Symbol & Beschreibung & Einheit\\ \midrule
	$\mathbf{A}$	& Systemmatrix linearisierten Modell &\\
	$\mathbf{A}^{*}$	& Systemmatrix reduziertes Modell &\\
	$\mathbf{B}$	& Eingangsmatrix linearisierten Modell& \\
	$\mathbf{B}^{*}$	& Eingangsmatrix reduziertes Modell &\\
	$\mathbf{C}$	& Ausgangsmatrix linearisierten Modell& \\
	$\mathbf{C}^{*}$	& Ausgangsmatrix reduziertes Modell &\\
	
	$\mathbf{C}_{x}$	& Corioliskraft-Matrix der $yz$-Ebene & \\
	
	$\mathbf{D}$	& Durchgangsmatrix linearisierten Modell& \\
	$\mathbf{D}^{*}$	& Durchgangsmatrix reduziertes Modell &\\
	
	$\mathbf{f}_{\text{NP,yz1}}$	& Kraftvektor des virtuellen Drehmomentes in $yz$-Ebene &  \unit{Nm} \\
	$\mathbf{f}_{\text{NP,yz2}}$	& Kraftvektor des Gegendrehmoments in der $yz$-Ebene &  \unit{Nm} \\
	$\mathbf{f}_{\text{NP,yz}}$	& Summe der einzelnen Kraftvektoren in  der $yz$-Ebene &  \unit{Nm} \\
	
	$\mathbf{G}_{x}$	& Gravitationskraft-Matrix der $yz$-Ebene &  \unit{N}\\
	
  $I_{\text{B,yz}}$ & Massentr�gheitsmoment K�rper in der $yz$-Ebene 			& \unit{kg\cdot m^{2}}\\
	$I_{\text{B,xz}}$ & Massentr�gheitsmoment K�rper in der $xz$-Ebene	 			& \unit{kg\cdot m^{2}}\\
	$I_{\text{B,xy}}$ & Massentr�gheitsmoment K�rper in der $xy$-Ebene				& \unit{kg\cdot m^{2}}\\
	
	$I_{\text{S}}$ 		& Massentr�gheitsmoment Ball 													& \unit{kg\cdot m^{2}}\\
	
	$I_{\text{W,yz}}$ & Massentr�gheitsmoment virtuelles Rad in der $yz$-Ebene	 			& \unit{kg\cdot m^{2}}\\
	$I_{\text{W,xz}}$ & Massentr�gheitsmoment virtuelles Rad in der $xz$-Ebene	 			& \unit{kg\cdot m^{2}}\\
	$I_{\text{W,xy}}$ & Massentr�gheitsmoment virtuelles Rad in der $xy$-Ebene	 			& \unit{kg\cdot m^{2}}\\
	
	$J$ 			& G�tema� f�r den linear-quadratischen Regelentwurf &\\
	$k_{\text{motor}}$		& Drehmomentkonstante & \unit{Nm/A}	\\
	$k_{\text{unit}}$		& Umrechungskonstante Drehmoment-Unit &	\unit{Nm/Unit}\\
	$\mathbf{\text{K}}$ & Matrix mit den Verst�rkungsfaktoren des Regler &\\
	
	$l$				& Pendell�nge & \unit{m} \\
	$l_{\text{pruef}}$		& L�nge Hebelarm des Pr�fhammers & \unit{m} \\
	$L$				& LANGRANGEsche Funktion & \unit{J} \\
	
	$\mathbf{M}_{\text{x}}$	& Massentr�gheitsmatrix der $yz$-Ebene & \\
	
	$m_{\text{B}}$ & Masse K�rper des Roboters 			& \unit{kg}\\
	$m_{\text{S}}$ & Masse Ball								 			& \unit{kg}\\
	$m_{\text{W}}$ & Masse virtuelles Rad			 			& \unit{kg}\\
	
	$r_{\text{B}}$ & Radius K�rper des Roboters			& \unit{m}\\
	$r_{\text{S}}$ & Radius Ball								 			& \unit{m}\\
	$r_{\text{W}}$ & Radius virtuelles Rad			 			& \unit{m}\\
	
	$\mathbf{\text{R}}$ & Gewichtungsmatrix der Stellgr��en &\\
	
	
	
	\end{tabularx}
	
	\begin{tabularx}{\textwidth}{@{}l@{\qquad}X@{\quad}p{18mm}}
	Symbol & Beschreibung & Einheit\\ \midrule
	$\mathbf{q}_{\text{yz,xz,xy}}$	& minimaler Koordinatenvektor der jeweiligen Ebene & \\
	
	$\mathbf{Q}$ 	& Gewichtungsmatrix der Zust�nde &\\
	
	$T_{\text{1,2,3}}$	& reale Drehmomente der einzelnen Motoren 	& \unit{Nm} \\
	$T_{\text{x,y,z}}$	& virtuelle Drehmomente 									 	& \unit{Nm} \\
	
	$T_{\text{B,yz}}$ & kinetische Energie K�rper des Roboters in der $yz$-Ebene &\unit{J} \\
	$T_{\text{S,yz}}$ & kinetische Energie Ball in der $yz$-Ebene 								&\unit{J} \\
	$T_{\text{W,yz}}$ & kinetische Energie virtuelles Rad in der $yz$-Ebene 			&\unit{J} \\
	
	$\mathbf{u}$ & Stellgr��envektor &\\
	
	$V_{\text{B,yz}}$ & potentielle Energie K�rper des Roboters in der $yz$-Ebene &\unit{J} \\
	$V_{\text{S,yz}}$ & potentielle Energie Ball in der $yz$-Ebene 							&\unit{J} \\
	$V_{\text{W,yz}}$ & potentielle Energie virtuelles Rad in der $yz$-Ebene 		&\unit{J} \\
	
	$\mathbf{x}$ & Zustandsvektor urspr�ngliches System &\\
	$\mathbf{x}^{*}$ & Zustandsvektor reduziertes System &\\
	
	$\dot{\mathbf{x}}$ & zeitliche Ableitung des Zustandsvektors des linearisierten Systems &\\
		$\dot{\mathbf{x}}_{nl}$ & zeitliche Ableitung des Zustandsvektors des urspr�nglichen nichtlinearen Systems &\\
	$\dot{\mathbf{x}}^{*}$ & zeitliche Ableitung des Zustandsvektors des reduzierten Systems &\\
	
	
	
	
\end{tabularx}
%
\paragraph*{Griechische Symbole und Formelzeichen}
\begin{tabularx}{\textwidth}{@{}l@{\qquad}X@{\quad}p{18mm}}
	Symbol & Beschreibung & Einheit\\ \midrule
  $\vartheta_{\text{x,y,z}}$ 	& Orientierung des Roboters 	& \unit{rad}\\
	$\varphi_{\text{x,y,z}}$ 		& Orientierung des Balles		 	& \unit{rad}\\
	$\psi_{\text{x,y,z}}$ 				& Orientierung des virtuellen Rades 	& \unit{rad}\\
	$\mathbf{\lambda}$ 		& Eigenwerte urspr�ngliches System 	&   \\
	$\mathbf{\lambda}^{*}$	& Eigenwerte reduziertes System 	&   \\
  
\end{tabularx}
%
\paragraph*{Abk�rzungen}
\begin{tabularx}{\textwidth}{@{}l@{\qquad}X}
K�rzel & vollst�ndige Bezeichnung  \\ \midrule
  CAD & Computer Aided Design \\
  CMU & Carnegie Mellon University \\
  EVA & Einlesen-Verarbeiten-Ausgeben\\
  ETHZ & Eidgen�ssische Technische Hochschule Z�rich \\
  IMU & inertiale Messeinheit\\
  LQR & linear-quadratische Regelung  \\
  ODE & Open Dynamics Engine \\
  PLA & Polylacide \\
  ROS & Robot Operating System \\
  
  
	
\end{tabularx}
%
\cleardoublepage

% =================================================================================
% Hauptteil
% =================================================================================
\pagenumbering{arabic}	% Hauptteil bekommt arabische Seitenzahlen