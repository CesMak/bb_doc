\documentclass[twoside,colorback,accentcolor=tud4c,11pt]{tudreport}
%\usepackage{ngerman}
\usepackage{german} %damit euro zeichen richtig
\usepackage{eurosym}
\usepackage[stable]{footmisc}
\usepackage[ngerman,pdfview=FitH,pdfstartview=FitV]{hyperref}
\usepackage{array}
\usepackage{booktabs}
\usepackage{multirow}
\usepackage{longtable}
\newcolumntype{L}[1]{>{\raggedright\let\newline\\\arraybackslash\hspace{0pt}}m{#1}}
\usepackage{listings}
\lstdefinestyle{BashInputStyle}{
	language=bash,
    basicstyle=\ttfamily,
	%numbers=left,
	%	numberstyle=\tiny,
	%	numbersep=3pt,
	frame=tb,
	frameshape={RYR}{Y}{Y}{RYR},
	columns=fullflexible,
	backgroundcolor=\color{gray!20},
	linewidth=0.95\linewidth,
	xleftmargin=0.1\linewidth,
  showspaces=false,
showtabs=false,
breaklines=true,
showstringspaces=false,
breakatwhitespace=true,
belowskip=1.2em,
aboveskip=1.2em,
} 

%defining a blue link:
\newcommand{\mylink}[2] {	\href{#1}{	\textit{\textcolor{blue}{#2}}}}


\newlength{\longtablewidth}
\setlength{\longtablewidth}{0.675\linewidth}

\title{Project Ballbot}
\subtitle{Markus Lamprecht \\ Florian M\"uller \\ Michael Suffel}
%\subsubtitle{email: \textaccent{tud-design@pro-kevin.de}}
%\uppertitleback{(\textaccent{\textbackslash uppertitleback})}
%\lowertitleback{(\textaccent{\textbackslash lowertitleback})\hfill\today}
\institution{TU Darmstadt, RTM}

\begin{document}
\maketitle
\tableofcontents


\chapter{Item - List}
\begin{tabular}{l l l l l}
	Item & \# & W.[g] & Weblink & Picture\\
	OpenCR Board (Controlling the motors, IMU)&1&60&\mylink{https://github.com/ROBOTIS-GIT/OpenCR/wiki/Hardware_Specification\#specification}{github\_wiki} 
	&\includegraphics[width=0.1\textwidth]{img/opencr.png}  \\
	
	
	UpBoard (Main PC)&1 &96 & \mylink{https://up-shop.org/up-boards/44-up-board-4gb-ram-64-gb-emmc.html}{\EUR{127}}
	&\includegraphics[width=0.1\textwidth]{img/upboard.jpg} \\
	
	Intel RealSense R200&1& 9.4& \mylink{https://www.intel.de/content/www/de/de/support/articles/000023534/emerging-technologies/intel-realsense-technology.html}{datasheet, \EUR{84.15}}&
	\includegraphics[width=0.1\textwidth]{img/r200.jpg} \\
	
	Laser Distance Sensor&1 &124 &\mylink{https://wiki.ros.org/hls_lfcd_lds_driver?action=AttachFile\&do=view\&target=LDS_Basic_Specification.pdf}{specs, \EUR{100}} & 
	\includegraphics[width=0.1\textwidth]{img/lasersensor.png}\\
	
	Battery: LI-PO 11.1 1800mAh LB-12 19&1&132 &\mylink{https://nodna.de/Robotis-LIPO-111V-Akkupack-1800mAh-LBS-012}{\EUR{44.90}} &
	\includegraphics[width=0.1\textwidth]{img/battery.png} \\
	
	Turtlebot3 Layers(125cmx125cm)&4& & & \\
	
	XM430-W350-R Dynamixel (Motors)&3&82 &\mylink{http://support.robotis.com/en/product/actuator/dynamixel_x/xm_series/xm430-w350.htm}{robotis,\EUR{250}} &
	\includegraphics[width=0.1\textwidth]{img/dynamixel.png} \\
	
	Ball(alum., dia.: 140mm, material thickness 2.5mm)&1&400&\mylink{http://www.ball-tech.de/Hohlkugeln/Aluminium/}{ball-tech gmbh,\EUR{40}. } & \includegraphics[width=0.1\textwidth]{img/ball.png}\\
	
	Omni wheels(dia: 60mm, thickness:25mm)&3&51.46 &\mylink{http://krause-robotics.de/xtshop/Antriebe/Raeder/Allseitenraeder/Allseitenraeder-60-mm:::99_100_106_114.html}{\EUR{10.38}} &  	\includegraphics[width=0.1\textwidth]{img/wheel.jpg}   \\
	
	Kreisring (PLA, 3D printeted)&1&28 & & 
	\includegraphics[width=0.1\textwidth]{img/kreisring.png}\\
	
	Halterung (PLA, 3D printeted)&3&18 & & 
	\includegraphics[width=0.1\textwidth]{img/halterung.png} \\
	
	Mitnehmer (PLA, 3D printeted)&3&8 & &
	\includegraphics[height=0.06\textwidth]{img/mitnehmer.png}  \\
	
	Plain washer (Beilagscheibe),(PLA, 3D printeted)&3&0.45 & &
	\includegraphics[height=0.06\textwidth]{img/beilagscheibe.png}  \\
	
	
		
		Omni double wheels(dia: 56mm, thickness:25mm)&3&62 &{\EUR{15}} &  	\includegraphics[width=0.05\textwidth]{img/wheel_double.jpg}   \\
		
		Mitnehmer double wheels&3&7 & &  	\includegraphics[width=0.05\textwidth]{img/mitnehmer_double.png}   \\
		
			Ball(alum., dia.: 140mm, material thickness 2.5mm)&1&400&\mylink{http://www.ball-tech.de/Hohlkugeln/Aluminium/}{ball-tech gmbh,\EUR{40}. } & \includegraphics[width=0.1\textwidth]{img/white_ball.png}\\
			
			Ball(gum., dia.: ??mm, material thickness ??mm)&1&326&link and cost&\includegraphics[width=0.1\textwidth]{img/yellow_ball.png}\\
			
	

	
	
	
\end{tabular}


\begin{tabular}{l l l l l}
	M3  (Mutter-Halterung-Kreisring-Layer)&9& & & \\
	M2.5  (Kreisring-Layer)&2& & & \\
	M3x8mm Halterung &6& &Zylinderkopf (Imbus) & \\
	M3x22mm Layer&3&1.34 &Zylinderkopf (Imbus) & \\
	M2.5x22 (Motoren-Halterung) &12&&Sechskant & \\
	M2.5x38 (Motoren-Rad)&3& &Zylinderkopf (Imbus) & \\
	M2.5x24 (Layer)&2& &Zylinderkopf (Imbus) & \\
	M2x6mm  (Mitnehmer-Motor)&12 & &Zylinderkopf (Imbus) & \\
	Distanzbolzen&???& &???& \\
\end{tabular}
% Please add the following required packages to your document preamble:
% \usepackage[normalem]{ulem}
% \useunder{\uline}{\ul}{}
\begin{table}[]
\centering
\caption{Screws:}
\label{my-label}
\begin{tabular}{|c|c|c|c|}
\hline
{\textbf{Type}} & {\textbf{Size}}                                             & { \textbf{Amount}} & { \textbf{Place}} \\ \hline
Cylinderhead screw  & M3 x 11mm                                                       & 8                     & Motor mounts         \\ \hline
Cylinderhead screw  & M2,5 x 22mm                                                     & 16                    & Motor plate          \\ \hline
Cylinderhead screw  & M2 x 6 mm                                                       & 18                    & Wheel shaft          \\ \hline
Cylinderhead screw  & \begin{tabular}[c]{@{}c@{}}M2,5  x 36 mm\\ (38 mm)\end{tabular} & 5                     & Wheel shaft cover    \\ \hline
Cylinderhead screw  & \begin{tabular}[c]{@{}c@{}}M3 x 20 mm\\ (21mm)\end{tabular}     & 4                     & Layer mounting       \\ \hline
Nut                 & M2                                                              & 5                     & Layer mounting       \\ \hline
Cylinderhead screw  & \begin{tabular}[c]{@{}c@{}}M2,5 x 22mm\\ (23mm)\end{tabular}    & 4                     & Layer mounting       \\ \hline
\end{tabular}
\end{table}


\textbf{Total Cost: \EUR{1176} + Cost of opencr board and all plastic (incl. tb3 structure) and scrwes }

\chapter{Simulation}

TODO: 
check if controller works

check why imu fails

\section{Launch}
These files are executed one after another:
\begin{enumerate}
	\item bb\_simulation: ballbot.launch
	\item bb\_description: bb\_description.launch
	\item bb\_description -> urdf: bb.xacro
	\item bb\_description -> urdf: bb.urdf.xacro
	\item bb\_description -> urdf: common\_properties.xacro
	\item bb\_description -> urdf: bb.gazebo.xacro
\end{enumerate}

\section{Gazebo - Controller Synchronization }
The calculation of the 2D torques takes around: 0.000187 s. 

Consider to take the right joint states as also all subwheel states are published.

Erst wenn das Motordrehmoment grosser als die wheel joint friction ist, bewegt sich das rad!.

Denke daran die joint states position und velocity in einen winkel und winkelgeschwindigkeit umzurechnen.

 The state of each joint (revolute or prismatic) is defined by:

 the position of the joint (rad or m),
 
 the velocity of the joint (rad/s or m/s) and 
 
 the effort that is applied in the joint (Nm or N).
 
 
 Muss ich nun in winkelgeschwindigkeit umrechnen oder nicht?!
 
 Das seltsame: Ich kann keinen kontinuierlichen effort draufgeben. Unabhängig von der publish rate, es wird of 3 mal eine 1 drauf gegeben und anschließend eine 0.
 
 Groesstes Problem bleibt: Die motor commands (torques) koennen nicht kontinuierlich rausgeschickt werden. Es werden immer wieder 0 rausgschickt ..... - warum ist das so?


use\_sim\_time parameter:
ros time is the same as simulation time when the use\_sim\_time parameter is enabled.

In order for a ROS node to use simulation time according to the /clock topic, the use\_sim\_timeparameter must be set to true before the node is initialized. This can be done in a launchfile or from the command line.

If the use\_sim\_time time parameter is set, the ROS Time API will return time=0 until it has received a value from the /clock topic. Then, the time will only be updated on receipt of a message from the /clock topic, and will stay constant between updates.

For calculations of time durations when using simulation time, clients should always wait until the first non-zero time value has been received before starting, because the first simulation time value from /clock topic may be a high value.

Note: Prior to ROS C Turtle, nodes were automatically subscribed to the /clock topic, and would use simulation time if there was anything published to the /clock topic. 

\section{Simulation design}

Ballbot SDF Reference:
\mylink{https://bitbucket.org/osrf/gazebo/issues/2335/how-to-set-the-friction-of-ballbot-the}{Ballbotmodel} 

We use not the sdf but the xacro description as in this example \mylink{http://gazebosim.org/tutorials/?tut=ros_urdf}{here}.

\begin{center}
		\includegraphics[width=0.2\textwidth]{img/filestructure.png} 
\end{center}

Gazebo uses different physics engines:\\
\begin{itemize}
	\item Open Dynamics Engine (ODE) (Default)
	\item Bullet
	\item Dynamic Animation and Robotics Toolkit (DART)
	\item Simbody
\end{itemize}
 which all have different friction etc. models.

Files:
\begin{itemize}
	\item bb.urdf.xacro: Link's: Visual description of the Robot and its collision model(STL file). Pose Mass and Inertias. Joint's: Pose,axis,effort and velocity limits, friction.
	\item common\_properties.xacro: Macros for color definition.
	\item bb.gazebo.xacro: gazebo references dynamics of the links: friction parameters (mu1,mu2), 
\end{itemize}

Gazebo Parameter's List:\\
\begin{tabular}{l L{12cm} l l}
	name(xacro)&description&value&sdf group\\
	mu1&is the Coulomb friction coefficient for the first friction direction&1.0&ode\\
	mu2& is the friction coefficient for the second friction direction (perpendicular to the first friction direction)&2.0&ode\\
	fdir1& 3-tuple specifying direction of mu1 in the collision local reference frame. fdir1 is the vector that defines the direction of mu1, which is the principal contact direction &0 0 0&ode\\
	kp&spring constant equivalents of a contact as a function of SurfaceParams::cfm and SurfaceParams::erp & &ode \\
	kd&spring damping constant equivalents of a contact as a function of SurfaceParams::cfm and SurfaceParams::erp.   & &ode \\
	cfm&Constraint Force Mixing parameter.& &ode \\
	erp&Error Reduction Parameter.& &ode \\
	min\_depth&Minimum depth before ERP takes effect.   & &ode \\
	max\_Vel& Maximum interpenetration error correction velocity.
	If set to 0, two objects interpenetrating each other will not be pushed apart.  & &ode \\
	slip1&Artificial contact slip in the primary friction direction  & &ode \\
	slip2&Artificial contact slip in the secondary friction dirction.& &ode \\
	\end{tabular}

See: \mylink{http://osrf-distributions.s3.amazonaws.com/gazebo/api/dev/classgazebo_1_1physics_1_1ODESurfaceParams.html}{ODESurfaceParams}

	Urdf Parameter's List: JOINT TAGS:\\
	\begin{tabular}{l l l}
		name(xacro tag)&description&value\\
		axis&this is the axis around the joint is revolting or linear &xyz=``0 1 0''\\
		dynamics&set the friction and the damping &friction=``0.7'' damping = ``0.0''\\
	\end{tabular}

\section{Gazebo Parameters}

\section{How to model an omni wheel gazebo:}
\begin{enumerate}
	\item \url{http://answers.gazebosim.org/question/5562/modeling-omni-wheels/}
	\item \url{http://answers.gazebosim.org/question/5562/modeling-omni-wheels/}
	\item \url{http://answers.gazebosim.org/question/5476/parameters-for-a-skid-steeringsimulated-tracked-robot-use-of-the-fdir1-tag/}
	\item \url{https://bitbucket.org/osrf/gazebo/pull-requests/2652/added-support-for-tracked-vehicles/diff}
	\item \url{https://bitbucket.org/osrf/gazebo/pull-requests/2652/added-support-for-tracked-vehicles/diff}
	\item \url{https://bitbucket.org/osrf/gazebo/issues/2068/directional-friction-still-broken}
	
	\item \url{https://answers.ros.org/question/212889/gazebo-planar-move-plugin-for-omni-directional-wheel/}
	
	
	For instance, the PR2 and Care-O-Bot are omnidirectional drive robots available for simulation in gazebo. Both use a system of four steered and driven casters (for a total of 8 motors) to achieve omnidirectional mobility. If you're interested in simulation of a meccanum-wheel drive robot, I'm not sure there is one available for gazebo. Last time I looked, the youbot for gazebo used no true meccanum wheels, but a similar system to the two robots I mentioned above.
\end{enumerate}

\section{omni wheel controllers}
\begin{enumerate}
	\item libgazebo\_ros\_skid\_steer\_drive.so \url{https://github.com/fsuarez6/labrob/blob/master/labrob_description/urdf/labrob.urdf.xacro}
\end{enumerate}
	
	\section{Equations for Controller:}
$\vartheta_{x,y,z}$ represent the orientation of the body

 $\varphi_{x,y,z}$ represent the orientation of the ball
 
  $\psi_{x,y,z}$ are the angles of the virtual actuating wheels.
  
$T_{x,y,z}$ are the virtual motor torques.
		
$T_{1,2,3}$ are the real motor torques.
	Inputs:
\begin{enumerate}
	\item The Gain Matrix K derived from the Simulink Simulation
	\item 	IMU-Measurements: $\vartheta_{x,y,z} (rad), \quad \dot{\vartheta}_{x,y,z} (rad/sec)$
	\item Motor-Measurements(rotation of actuating wheel): $\psi_{x,y,z} (rad), \quad \dot{\psi}_{x,y,z} (rad/sec)$
	\end{enumerate}	
	
	Wie von virtual wheels auf real wheels umrechnen?!
	
	

	

	\section{Control}
	sobald diff drive plugin angeschaltet drehen sich die raeder viel zu schnell ....
	
	Diff Drive in ballbot.launch an oder ausschalten. 
	
	in bb.gazebo.xacro transmission und controller festlegen.
	
	zudem yaml file(currently I use: effort\_controllers/JointVelocityController)
	
	Effort Joint Interface as Hardware Interface is used.
	
	Do this example first: \url{http://gazebosim.org/tutorials/?tut=ros_control}
	
	Also try this bb8 gazebo tutorial: \url{https://www.youtube.com/watch?v=j5qC9l448p8}
	
	
	
	\subsection{Plugins}
	\begin{itemize}
		\item gazebo-ros-control
		\item diff drive
	\end{itemize}
	
	\subsection{Launch}
	\begin{lstlisting}[style=BashInputStyle]
	roslaunch rrbot_control rrbot_control.launch
	\end{lstlisting}
	
	These files are executed one after another:
	\begin{enumerate}
		\item load config
		\item controller\_spawner
	\end{enumerate}
	
	\section{Sensors}
	\subsection{IMU}
	
		We want to simulate the IMU of the opencr board. STRG+T to see imu topic values!
	\mylink{https://www.youtube.com/watch?v=wXN_7oRHst0}{Imu of opencr board simulated}

	Simulate like this:
	rviz rviz dann als fixed frame nimm: imu\_link. Und add topic imu und waehle als topic ballbot/sensor/imu
	
	The simulated IMU outputs values like: orientation (x,y,z,w), angluar velocity(x,y,z), linear velocity(x,y,z), linear acceleration(x,y,z).
	
	The opencr real IMU gives values like: orientation(x,y,z,w), angular velocity(x,y,z), linear acceleration(x,y,z) see \mylink{OpenCR board IMU }{http://turtlebot3.readthedocs.io/en/latest/appendix\_opencr.html}
	
	\chapter{Model}
	
	\section{Composition}
	The Ballbot consists of three parts, which are depicted in Figure \ref{fig:Structure}.
	
	\begin{itemize}
		\item Body with motors
		\item 3 omni-directional wheels
		\item Ball
	\end{itemize}
	
	
	\begin{figure}[htbp]
		\centering
		\includegraphics[width=0.9\textwidth]{img/Strucuture.PNG}
		\caption{Parts for the 3D-Model}
		\label{fig:Structure}
	\end{figure}
	
	\section{Assumptions}
	
	To reduce the complexity of the system, the following assumptions are made: 
	
	\begin{itemize}
		\item \textbf{No slip} between the contact points between the ball/ground and wheels/ball
		\item \textbf{No friction}; except the friction, which occurs at the rotation of the ball around the z-axis
		\item \textbf{No deformation}
		\item \textbf{Fast motor dynamics}; The controlling of the motor is much faster than the controller of the Ballbot
		\item \textbf{Ball moves only horizontal}
	\end{itemize}
			
			
			\section{TODO}
			
			\begin{enumerate}
				\item Horn komplett rein auf beilagscheibe und schauen dass das mit der 0 position stimmt!
			\end{enumerate}
	\section{Model Parameters}

	
% Please add the following required packages to your document preamble:
% \usepackage[normalem]{ulem}
% \useunder{\uline}{\ul}{}
\begin{table}[]
\centering
\caption{My caption}
\label{my-label}
\begin{tabular}{cccc}
\hline
\multicolumn{1}{|c|}{{ \textbf{Parameter}}}                                                                          & \multicolumn{1}{c|}{{\textbf{Variable}}} & \multicolumn{1}{c|}{{ \textbf{Value}}}                  & \multicolumn{1}{c|}{{ \textbf{Source}}} \\ \hline
                                                                                                                        &                                              &                                                            &                                            \\ \hline
\multicolumn{1}{|c|}{Mass of the ball}                                                                                  & \multicolumn{1}{c|}{$m_{K}$}                & \multicolumn{1}{c|}{0,4 kg}                                & \multicolumn{1}{c|}{Datasheet}             \\ \hline
\multicolumn{1}{|c|}{Mass of the ball}                                                                                  & \multicolumn{1}{c|}{$m_{K}$}                & \multicolumn{1}{c|}{0,397 kg}                              & \multicolumn{1}{c|}{Measured}              \\ \hline
\multicolumn{1}{|c|}{Mass of the ball}                                                                                  & \multicolumn{1}{c|}{$m_{K}$}                & \multicolumn{1}{c|}{0,631 kg}                              & \multicolumn{1}{c|}{SolidEdge}             \\ \hline
\multicolumn{1}{|c|}{\begin{tabular}[c]{@{}c@{}}Mass of the body, complete\\ (with motors/wheels)\end{tabular}}         & \multicolumn{1}{c|}{$m_{B}$}                & \multicolumn{1}{c|}{?}                                     & \multicolumn{1}{c|}{Measured}              \\ \hline
\multicolumn{1}{|c|}{\begin{tabular}[c]{@{}c@{}}Mass of the body, complete\\ (with motors/wheels)\end{tabular}}         & \multicolumn{1}{c|}{$m_{B}$}                & \multicolumn{1}{c|}{1,785 kg}                              & \multicolumn{1}{c|}{SolidEdge}             \\ \hline
\multicolumn{1}{|c|}{\begin{tabular}[c]{@{}c@{}}Mass of the body\\ (without motors/wheels)\end{tabular}}                & \multicolumn{1}{c|}{$m_{B}$}                & \multicolumn{1}{c|}{?}                                     & \multicolumn{1}{c|}{Measured}              \\ \hline
\multicolumn{1}{|c|}{\begin{tabular}[c]{@{}c@{}}Mass of the body\\ (without motors/wheels)\end{tabular}}                & \multicolumn{1}{c|}{$m_{B}$}                & \multicolumn{1}{c|}{1,394 kg}                              & \multicolumn{1}{c|}{SolidEdge}             \\ \hline
\multicolumn{1}{|c|}{Mass of Omniwheel}                                                                                 & \multicolumn{1}{c|}{$m_{OW}$}               & \multicolumn{1}{c|}{0,050 kg}                              & \multicolumn{1}{c|}{Measured}              \\ \hline
\multicolumn{1}{|c|}{Mass of Omniwheel}                                                                                 & \multicolumn{1}{c|}{$m_{OW}$}               & \multicolumn{1}{c|}{0,046 kg}                              & \multicolumn{1}{c|}{SolidEdge}             \\ \hline
\multicolumn{1}{|c|}{Mass of the virtual wheel}                                                                         & \multicolumn{1}{c|}{$m_{VW}$}               & \multicolumn{1}{c|}{0,384 kg}                              & \multicolumn{1}{c|}{Measured}              \\ \hline
\multicolumn{1}{|c|}{\begin{tabular}[c]{@{}c@{}}Mass of the substructure, complete\\ (with motors/wheels)\end{tabular}} & \multicolumn{1}{c|}{$m_{S}$}                & \multicolumn{1}{c|}{0,506 kg}                              & \multicolumn{1}{c|}{Measured}              \\ \hline
\multicolumn{1}{|c|}{\begin{tabular}[c]{@{}c@{}}Mass of the substructure, complete\\ (with motors/wheels)\end{tabular}} & \multicolumn{1}{c|}{$m_{S}$}                & \multicolumn{1}{c|}{0,457 kg}                              & \multicolumn{1}{c|}{SolidEdge}             \\ \hline
\multicolumn{1}{|c|}{Mass of plate}                                                                                     & \multicolumn{1}{c|}{$m_{P}$}                & \multicolumn{1}{c|}{0,078 kg}                              & \multicolumn{1}{c|}{Measured}              \\ \hline
\multicolumn{1}{|c|}{Mass of plate}                                                                                     & \multicolumn{1}{c|}{$m_{P}$}                & \multicolumn{1}{c|}{?}                                     & \multicolumn{1}{c|}{SolidEdge}             \\ \hline
                                                                                                                        &                                              &                                                            &                                            \\ \hline
\multicolumn{1}{|c|}{Radius of the ball}                                                                                & \multicolumn{1}{c|}{$r_{K}$}                & \multicolumn{1}{c|}{0,07 m}                                & \multicolumn{1}{c|}{Datasheet}             \\ \hline
\multicolumn{1}{|c|}{Radius of the body}                                                                                & \multicolumn{1}{c|}{$r_{B}$}                & \multicolumn{1}{c|}{0,0703 m}                              & \multicolumn{1}{c|}{Measured}              \\ \hline
\multicolumn{1}{|c|}{Radius of the Wheels}                                                                              & \multicolumn{1}{c|}{$r_{W}$}                & \multicolumn{1}{c|}{0,03 m}                                & \multicolumn{1}{c|}{Datasheet}             \\ \hline
\multicolumn{1}{|c|}{Height of the center of gravity}                                                                   & \multicolumn{1}{c|}{$l$}                       & \multicolumn{1}{c|}{0,24045 m}                             & \multicolumn{1}{c|}{SolidEdge}             \\ \hline
\multicolumn{1}{|c|}{Height of the body}                                                                                & \multicolumn{1}{c|}{$h$}                       & \multicolumn{1}{c|}{0,34294 m}                             & \multicolumn{1}{c|}{SolidEdge}             \\ \hline
\multicolumn{1}{|c|}{Inertia of the Ball}                                                                               & \multicolumn{1}{c|}{$\Theta_{K}$}           & \multicolumn{1}{c|}{0,00131 $kgm^{2}$}     & \multicolumn{1}{c|}{Computed}              \\ \hline
\multicolumn{1}{|c|}{Inertia of the Body (x-axis)}                                                                      & \multicolumn{1}{c|}{$\Theta_{Bx}$}          & \multicolumn{1}{c|}{0,08751 $kgm^{2}$}     & \multicolumn{1}{c|}{SolidEdge}             \\ \hline
\multicolumn{1}{|c|}{Inertia of the Body (y-axis)}                                                                      & \multicolumn{1}{c|}{$\Theta_{By}$}          & \multicolumn{1}{c|}{0,08788 $kgm^{2}$}     & \multicolumn{1}{c|}{SolidEdge}             \\ \hline
\multicolumn{1}{|c|}{Inertia of the body (z-axis)}                                                                      & \multicolumn{1}{c|}{$\Theta_{Bz}$}          & \multicolumn{1}{c|}{0,00329 $kgm^{2}$}     & \multicolumn{1}{c|}{SolidEdge}             \\ \hline
\multicolumn{1}{|c|}{Inertia of the body (xy plane)}                                                                    & \multicolumn{1}{c|}{$\Theta_{Bxy}$}         & \multicolumn{1}{c|}{-0,00001 $kgm^{2}$}    & \multicolumn{1}{c|}{SolidEdge}             \\ \hline
\multicolumn{1}{|c|}{Inertia of the body (xz plane)}                                                                    & \multicolumn{1}{c|}{$\Theta_{Bxz}$}         & \multicolumn{1}{c|}{0,00203 $kgm^{2}$}     & \multicolumn{1}{c|}{SolidEdge}             \\ \hline
\multicolumn{1}{|c|}{Inertia of the body(zy plane)}                                                                     & \multicolumn{1}{c|}{$\Theta_{Bzy}$}         & \multicolumn{1}{c|}{0,00018 $kgm^{2}$}     & \multicolumn{1}{c|}{SolidEdge}             \\ \hline
\multicolumn{1}{|c|}{Inertia of the rotor (motor)}                                                                      & \multicolumn{1}{c|}{$\Theta_{M}$}           & \multicolumn{1}{c|}{0,444e-6 $kgm^{2}$}    & \multicolumn{1}{c|}{Adoption}              \\ \hline
\multicolumn{1}{|c|}{Inertia of Omniwheel}                                                                              & \multicolumn{1}{c|}{$\Theta_{OW}$}          & \multicolumn{1}{c|}{0.000023157 $kgm^{2}$} & \multicolumn{1}{c|}{Computed}              \\ \hline
\multicolumn{1}{|c|}{Inertia of the actuating wheel in yz/xz}                                                            & \multicolumn{1}{c|}{$\Theta_{W}$}           & \multicolumn{1}{c|}{0,058873 $kgm^{2}$}    & \multicolumn{1}{c|}{Computed}              \\ \hline
\multicolumn{1}{|c|}{Inertia of the actuating wheel in xy}                                                              & \multicolumn{1}{c|}{$\Theta_{Wxy}$}         & \multicolumn{1}{c|}{0,16656 $kgm^{2}$}     & \multicolumn{1}{c|}{Computed}              \\ \hline
                                                                                                                        &                                              &                                                            &                                            \\ \hline
\multicolumn{1}{|c|}{Gear ratio}                                                                                        & \multicolumn{1}{c|}{$i$}                       & \multicolumn{1}{c|}{353,5}                                 & \multicolumn{1}{c|}{Datasheet}             \\ \hline
\multicolumn{1}{|c|}{Gravitational acceleration}                                                                        & \multicolumn{1}{c|}{$g$}                       & \multicolumn{1}{c|}{9,81 $m/s^{2}$}        & \multicolumn{1}{c|}{BachelorThesis}        \\ \hline
\end{tabular}
\end{table}

Traegheitsmoment kugel Hohlzylinder: $J=m \frac{r_i^2+r_a^2}{2} = 0.4 * \frac{65^2+70^2}{2} kg *10^ {-6} m^2=1.825 *10^{-3} kg m^2$ 

Traegheitsmoment wheel Vollzylinder: $J=m \frac{1}{2} r^2 = 0.05146 * 0.020^2 = 1.0292 * 10^{-4} kg m^2$ 

\end{document}
